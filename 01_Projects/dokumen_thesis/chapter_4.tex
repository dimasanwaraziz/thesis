\chapter{PRESENTATION, ANALYSIS AND INTERPRETATION OF DATA}
In thesis writing, the most difficult part to defend is chapter 4 because it is in this section where you will present the results of the whole study. Here is a sample thesis format.

% =========================================================
\section{Presentation of Data}
% =========================================================
Present the findings of the study in the order of the specific problem as stated in the statement of the Problem. Present the data in these forms: (a) tabular; (b) textual; and (c) graphical (optional). The ZOOM LENS approach may be used for purposes of clarity in the presentation of data, i.e. general to particular, macro to micro or vice versa. \textbf{(Note: Mean of data in here is data of the experiment result, not the data for input of the system)}
% =========================================================
\section{Analysis of the Data}
% =========================================================
Data may be analyzed quantitatively or qualitatively depending on the level of measurement and the number of dimensions and variables of the study.
Analyze in depth to give meaning to the data presented in the data presented in the table. Avoid table reading.
State statistical descriptions in declarative sentences, e.g. in the studies involving:
\begin{enumerate}
	\item Correlation
	\begin{enumerate}
		\item State level of correlation
		\item State whether positive or negative
		\item Indicate the level of significance
		\item Make a decision
	\end{enumerate}
	\item Differences of Measures
	\begin{enumerate}
		\item State the obtained statistical results
		\item Indicate the level of significance of the difference
		\item Make a decision
	\end{enumerate}
	\item Interpretation of Data
	\begin{enumerate}
		\item Establish interconnection between and among data
		\item Check for indicators whether hypothesis/es is/are supported or not by findings.
		\item Link the present findings with the previous literature.
		\item Use parallel observations with contemporary events to give credence presented in the introduction.
		\item Draw out implications.
	\end{enumerate}
\end{enumerate}

In thesis writing, the Chapter is simply a summary of what the researcher had done all throughout the whole research. 
% =========================================================
\section{Summary of Findings}
% =========================================================
This describes the problem, research design, and the findings (answer to the questions raised). The recommended format is the paragraph form instead of the enumeration form. For each of the problems, present: (a) the salient findings; and (b) the results of the hypothesis tested.
