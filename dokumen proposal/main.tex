\documentclass{ittelkom}
\usepackage{graphicx}
\usepackage{multirow}
\usepackage{float} 
\usepackage{tabularx, ragged2e, lmodern, bigstrut}
\usepackage[sgvnames, table]{xcolor}
\newcommand{\blue}{\cellcolor{blue!75}}
\usepackage[style=numeric,sorting=none]{biblatex}
\addbibresource{referensi.bib}

\begin{document}

\project{Multiple Embeding Steganography for RGB Image}
\submit{Dimas Anwar Aziz}
\nim{203022410012}
\concentration{Cyber Security}
\email{dimasanwaraziz@student.telkomuniversity.ac.id}
\supervisor{Prof.\ Ari Moesriami Barmawi, Ph.D.}
\submitdate{08}{10}{2024}

\summary{Topic Summary}
{This is a summary and a word length of 300 words. Summary is written briefly from the entire contents of the thesis/proposal and so on until it is finished.}

\newpage
\section{INTRODUCTION \color{red} (assessment 1)}
Steganografi merupakan teknik menyembunyikan pesan di dalam suatu media penyisipan pesan atau cover image,
sehingga keberadaan pesan rahasia yang disisipkan tidak dapat dilihat secara langsung \cite{pujianto2021uji}. Steganografi
memungkinkan pertukaran pesan rahasia melalui penyembunyian informasi pada berbagai media digital seperti gambar, video, dan audio,
tanpa menimbulkan kecurigaan \cite{siaulhak2023sistem}. Steganografi dapat dipandang sebagai kelanjutan
kriptografi dan dalam prakteknya pesan rahasia dienkripsi terlebih dahulu, kemudian
cipherteks disembunyikan di dalam media lain sehingga pihak ketiga tidak menyadari
keberadaannya. Pesan rahasia yang disembunyikan dapat diekstraksi kembali persis
sama seperti aslinya \cite{soetarmono2012studi}.

Di antara berbagai media digital yang dapat digunakan untuk steganografi,
gambar menjadi pilihan yang populer karena beberapa alasan. Pertama, gambar
digital tersedia secara luas dan pertukaran gambar di internet adalah hal yang
umum, sehingga tidak menimbulkan kecurigaan. Kedua, gambar memiliki kapasitas
untuk menyembunyikan informasi. Ketiga, manipulasi pixel pada gambar dapat
dilakukan dengan berbagai teknik. Namun, tantangan utama dalam steganografi
gambar adalah menemukan keseimbangan antara kapasitas penyembunyian data,
kualitas gambar, dan ketahanan terhadap deteksi \cite{fikri2022optimasi}.

Penelitian sebelumnya telah menunjukkan kemajuan signifikan dalam meningkatkan
kapasitas penyembunyian data pada gambar grayscale. Kingsley et al.
\cite{kingsley2020improving} berhasil meningkatkan kapasitas penyembunyian data
hingga 450\% menggunakan metode code base. Meski begitu, penggunaan gambar
grayscale membatasi aplikasi praktis karena hanya memiliki satu channel
pencahayaan. Di sisi lain, Fikri et al. \cite{fikri2022optimasi}
mendemonstrasikan bahwa penerapan steganografi pada gambar berwarna memiliki
ketahanan yang baik, meskipun gambar dengan dominasi warna hitam menunjukkan
kompatibilitas yang kurang optimal. Berdasarkan temuan-temuan tersebut,
penelitian ini bertujuan untuk memperluas metode embedding ke gambar berwarna
yang memiliki tiga channel warna (Merah, Hijau, dan Biru). Fokus utama adalah
untuk meningkatkan kapasitas penyembunyian data sambil mempertahankan kualitas
visual dan ketahanan terhadap deteksi. Penelitian ini juga akan mengevaluasi
kompatibilitas metode yang diusulkan dengan berbagai jenis gambar berwarna,
termasuk yang memiliki area gelap yang luas. Dengan mengoptimalkan penggunaan
ketiga channel warna, diharapkan dapat dicapai peningkatan signifikan dalam
kapasitas penyembunyian data dibandingkan dengan metode yang hanya menggunakan
gambar grayscale, sambil tetap mempertahankan kealamiahan gambar yang
dihasilkan.

Penelitian ini mencakup empat aspek utama: pengetahuan, pengguna, kegunaan, dan
infrastruktur. Aspek pengetahuan meliputi teknik steganografi gambar berwarna,
metode berbasis kode, dan analisis kualitas gambar digital, termasuk manipulasi
channel warna RGB. Pengguna potensial meliputi profesional keamanan informasi,
peneliti, dan pengembang aplikasi keamanan data. Kegunaan mencakup implementasi
dalam sistem data hiding, perlindungan hak cipta, dan watermarking, dengan
fokus pada integrasi yang mudah, efisiensi, dan keseimbangan antara kapasitas
penyembunyian data, kualitas gambar, dan ketahanan terhadap deteksi. Aspek
infrastruktur melibatkan penggunaan Mac mini M2 dengan spesifikasi prosesor
Apple M2 chip dengan 8-core CPU dan 10-core GPU, memori 8GB, penyimpanan SSD
256GB, dan sistem operasi macOS.

\section{Preliminary Literature Review \color{red}(assessment 2)}
Kingsley et al (2020) pada papernya membahas tentang rendahnya kapasitas penyisipan dan nilai PSNR (Peak Signal-to-Noise Ratio) pada skema steganografi berbasis kode yang ada, yang hanya mencapai 150\% kapasitas dan 48 dB kualitas visual gambar stegano. Untuk mengatasi masalah ini, metode yang diusulkan adalah teknik penyisipan ganda (multiple embedding), yang bertujuan untuk menyisipkan bit rahasia lebih dari sekali pada LSB (Least Significant Bit) dari piksel yang dipilih berdasarkan kunci rahasia. Parameter yang digunakan untuk mengukur keberhasilan metode ini meliputi kapasitas penyisipan dan nilai PSNR dari gambar stego yang dihasilkan. Kelebihan dari metode ini adalah kemampuan untuk mencapai kapasitas penyisipan yang lebih tinggi hingga 450\% dan nilai PSNR yang meningkat menjadi 51 dB, serta meningkatkan ketahanan terhadap serangan seperti kebisingan dan kompresi JPEG. Namun, penelitian ini masih diterapkan pada gambar hitam putih \cite{kingsley2020improving}.

Pada penelitian yang dilakukan oleh Dwi Andika et al (2020), mereka membahas
keterbatasan kapasitas penyimpanan dalam teknik steganografi menggunakan
algoritma GifShuffle pada citra GIF. Tantangan utama yang dihadapi adalah
kesulitan dalam menyisipkan pesan teks dengan ukuran besar. Untuk mengatasi
masalah ini, mereka memodifikasi nilai bit dalam penyimpanan pesan, yang
bertujuan meningkatkan kapasitas tanpa mengorbankan kualitas visual gambar. Uji
coba menggunakan parameter seperti ukuran data yang dapat disisipkan serta
kualitas gambar menunjukkan bahwa metode ini berhasil meningkatkan kapasitas
penyimpanan hingga lebih dari 256 KB tanpa mengurangi kualitas gambar secara
signifikan. Namun, metode ini memerlukan modifikasi tambahan pada algoritma
GifShuffle yang dapat meningkatkan kompleksitas implementasi dan pemrosesan
\cite{andika2020modifikasi}.

Susanto et al (2020) dalam penelitiannya menggabungkan metode steganografi
Least Significant Bit (LSB) dengan algoritma enkripsi RSA untuk meningkatkan
keamanan dalam penyisipan pesan terenkripsi pada gambar. Parameter yang
digunakan untuk mengukur kualitas gambar setelah penyisipan adalah Peak
Signal-to-Noise Ratio (PSNR), dengan hasil yang menunjukkan PSNR tertinggi
sebesar 78 dB untuk pesan berukuran 1024 bit. Kelebihan metode ini adalah
perubahan kualitas gambar yang hampir tidak terlihat, namun kekurangannya
adalah kompleksitas yang meningkat seiring dengan ukuran pesan dan besarnya
kunci RSA yang digunakan, yang memengaruhi performa
\cite{susanto2020kombinasi}.

Basri et al (2021) meneliti penggunaan metode steganografi dengan teknik Least
Significant Bit (LSB) untuk menyembunyikan gambar di dalam gambar lainnya,
khususnya dalam konteks interaksi sosial melalui media digital. Penelitian ini
mengukur rasio ukuran gambar tersembunyi terhadap gambar cover serta kemampuan
gambar cover mempertahankan kualitas visualnya. Hasilnya menunjukkan bahwa
metode ini efektif menyembunyikan informasi tanpa perubahan signifikan pada
gambar cover. Namun, metode ini memiliki keterbatasan dalam menangani gambar
dengan transparansi, karena hanya memperhitungkan komponen warna merah, hijau,
dan biru (RGB), tanpa memperhitungkan komponen alpha \cite{basri2021penerapan}.

Penelitian Wiranata et al (2021) berfokus pada penyisipan pesan rahasia dalam
gambar dan audio menggunakan metode Least Significant Bit (LSB) yang
dikombinasikan dengan enkripsi Caesar Chipper dan Rivest Code 4 untuk menjaga
kerahasiaan data. Penelitian ini menguji kemampuan aplikasi untuk
menyembunyikan dan mengambil pesan secara utuh, serta perubahan ukuran file
gambar dan audio setelah proses penyisipan. Kelebihan metode ini adalah
kemampuannya menjaga kerahasiaan tanpa perubahan signifikan pada kualitas
gambar atau suara. Namun, ada perubahan ukuran file yang diakibatkan oleh
proses enkripsi dan penyisipan pesan \cite{wiranata2021aplikasi}.

% fasilitas belum, terkait perangkat simulasi. Masukan perangkat yang digunakan.

Penelitian yang dilakukan oleh Abdillah et al (2023) membahas tentang
perlindungan data dan akses dengan menggunakan steganografi melalui teknik
Least Significant Bit (LSB), di mana teks disisipkan dalam gambar melalui
perubahan nilai pixel terkecil. Parameter yang diukur mencakup akurasi
penyisipan teks dan kualitas gambar yang tetap terjaga setelah proses encoding
dan decoding. Kelebihan utama metode ini adalah kemampuannya menyembunyikan
informasi tanpa mengubah kualitas visual gambar secara signifikan. Namun,
metode ini memiliki keterbatasan dalam kapasitas penyisipan teks dan rentan
terhadap serangan manipulasi gambar yang dapat merusak data tersembunyi
\cite{abdillah2023implementasi}.

\section{Problem Statement \textcolor{red}{(Assessment 2)}} \label{request}

Penelitian sebelumnya dalam steganografi berbasis kode telah menunjukkan
kemajuan signifikan. Kingsley et al. \cite{kingsley2020improving} berhasil
meningkatkan kapasitas penyembunyian data dari sekitar 150\% menjadi 450\%,
yang merupakan peningkatan substansial untuk case yang membutuhkan kapasitas
tinggi. Meskipun demikian, penelitian pada gambar grayscale
\cite{soetarmono2012studi} masih membatasi pengaplikasiannya karena kurangnya
fleksibilitas dalam penggunaan channel warna.

Sementara itu, penerapan steganografi pada gambar berwarna (RGB) membuka
peluang untuk peningkatan kapasitas, namun juga menghadirkan tantangan baru.
Fikri et al. \cite{fikri2022optimasi} menemukan bahwa meskipun gambar RGB
menawarkan robbustnes yang baik, terdapat masalah kompatibilitas pada gambar
dengan dominasi warna gelap. Selain itu, penggunaan multiple embedding pada
gambar RGB berpotensi meningkatkan kompleksitas komputasi
\cite{juneja2013improved}.

\section{Objective and Hypothesis \color{red}(assessment 3)} \label{hyp}

Penelitian ini memiliki dua tujuan utama beserta hipotesis terkait. Pertama,
mengembangkan teknik steganografi berbasis kode dengan multiple embedding untuk
gambar RGB dan meningkatkan kapasitas penyembunyian data minimal 50\%
dibandingkan dengan metode sebelumnya, dengan harapan bahwa metode ini akan
meningkatkan kapasitas penyembunyian data secara signifikan dan penerapan pada
gambar RGB akan menghasilkan peningkatan kapasitas lebih besar dibandingkan
pada gambar grayscale. Kedua, mempertahankan PSNR di atas 40 dB untuk gambar
stego, dengan hipotesis bahwa peningkatan kapasitas melalui multiple embedding
tidak akan menurunkan kualitas visual secara signifikan
\cite{nasution2018image}.

% \subsection{Tujuan Utama Penelitian}
% \begin{enumerate}
%     \item Mengembangkan teknik steganografi berbasis kode dengan multiple embedding untuk gambar RGB.
%     \item Meningkatkan kapasitas penyembunyian data minimal 50\% dibanding metode konvensional.
%     \item Mempertahankan PSNR di atas 40 dB untuk gambar stego.
%     \item Mengevaluasi ketahanan metode terhadap serangan steganalisis.
% \end{enumerate}

% \subsection{Hipotesis yang Diajukan}
% \begin{itemize}
%     \item H1: Multiple embedding akan meningkatkan kapasitas penyembunyian data secara signifikan dibanding metode steganografi berbasis kode konvensional.
%     \item H2: Peningkatan kapasitas melalui multiple embedding tidak akan menurunkan kualitas visual (PSNR $>$ 40 dB) secara signifikan.
%     \item H3: Metode yang diusulkan akan lebih tahan terhadap steganalisis dibanding steganografi LSB konvensional.
%     \item H4: Penerapan pada gambar RGB akan menghasilkan peningkatan kapasitas lebih besar dibanding pada gambar grayscale.
% \end{itemize}

%\begin{figure}
%\centering
%\includegraphics[width=0.4\textwidth]{TelU_1.jpg}
%%\caption{Logo University}
%\end{figure}

\section{Research Method \color{red}(assessment 4,5)}
Penelitian ini akan menggunakan pendekatan eksperimental untuk menguji hipotesis yang diajukan. Metode penelitian mencakup beberapa tahap utama, mengintegrasikan konsep dasar steganografi dan mengatasi tantangan khusus steganografi berbasis kode dengan teknik penyisipan ganda.

\subsection{Identifikasi Kebutuhan}

Berdasarkan analisis literatur dan tujuan penelitian, kebutuhan utama untuk
pengembangan metode steganografi berbasis kode dengan penyisipan ganda:

\begin{enumerate}
    \item Algoritma penyisipan dengan multiple embedding pada gambar RGB
    \item Peningkatan kapasitas penyembunyian data minimal 50\%
    \item Teknik mempertahankan kualitas visual (PSNR $>$ 40 dB)
\end{enumerate}

Identifikasi kebutuhan ini akan menjadi dasar untuk proses desain dan
implementasi sistem steganografi yang diusulkan.

\subsection{Proses Desain}
Berdasarkan kebutuhan yang diidentifikasi, penelitian ini akan merancang
algoritma untuk steganografi berbasis kode dengan multiple embedding.

\begin{figure}[h]
    \centering
    \includegraphics[width=0.8\textwidth]{gambar/proposed.png}
    \caption{Design method}
    \label{fig:flow}
\end{figure}

% jelaskan juga metode di bab ini
% Cari metode untuk handle gambar berwarna ini
% Rencana metodenya apa, tulis lagi per prosesnya. Cari reasoning, problemnya harus disolusikan.

Arsitektur sistem umum akan mencakup

\begin{enumerate}
    \item \textbf{Secret Image}

          Secret Image merupakan informasi rahasia yang akan disembunyikan dalam sistem
          steganografi, yang dapat mencakup gambar dalam berbagai format (JPG, PNG) yang
          akan dilindungan keamanan dan kerahasiaan. Pemilihan jenis data rahasia ini
          akan mempengaruhi kapasitas penyimpanan yang diperlukan dan metode penyisipan
          yang digunakan dalam proses steganografi.

          Gambar ini merupakan inputan dinamis user untuk nanti disisipkan pada cover
          image yang berwarna, dan nanti akan dicek terkait berapa kapasitas dan noise
          yang dihasilkan dari hasil penyisipan secret image ini.

    \item \textbf{Encoding Process}

          Proses \textit{encoding} bertujuan untuk menyisipkan pesan (teks) ke dalam
          gambar dengan menggunakan \textbf{Least Significant Bit (LSB)} dari setiap
          komponen warna pixel (R, G, B).

          Berikut adalah penjelasan langkah-langkah proses encoding:

          \begin{enumerate}
              \item \textbf{Persiapan Data}

                    \begin{itemize}
                        \item \textbf{Input:}
                              \begin{itemize}
                                  \item Gambar yang akan digunakan sebagai media penyisipan.
                                  \item Pesan teks yang ingin disembunyikan.
                              \end{itemize}

                        \item \textbf{Tambahan Terminator:}
                              Pesan diberi terminator (\texttt{\textbackslash 000}) di akhir sebagai penanda bahwa pesan telah selesai.
                              \begin{itemize}
                                  \item Contoh:
                                        \begin{quote}
                                            \textbf{Pesan awal:} "Hello, World!"
                                        \end{quote}
                                        \begin{quote}
                                            \textbf{Pesan yang akan diproses:} "Hello, World!\texttt{\textbackslash 000}"
                                        \end{quote}
                              \end{itemize}

                        \item \textbf{Konversi ke Bit:}
                              Setiap karakter diubah menjadi 8 bit (format byte).
                              \begin{itemize}
                                  \item Contoh huruf "H" (ASCII 72): 01001000.
                              \end{itemize}
                    \end{itemize}

              \item \textbf{Iterasi Gambar Pixel-per-Pixel}

                    Gambar diakses pixel-per-pixel, dan untuk setiap pixel:
                    \begin{itemize}
                        \item Komponen warna R (Red), G (Green), dan B (Blue) diekstrak.
                        \item Setiap komponen warna memiliki nilai 8 bit (0-255).
                    \end{itemize}

              \item \textbf{Menyisipkan Bit ke LSB}

                    Dari setiap byte pesan (8 bit), bit pesan disisipkan satu per satu ke dalam LSB
                    (Least Significant Bit) dari komponen R, G, dan B. LSB adalah bit paling rendah
                    dari nilai komponen warna. Mengubah LSB tidak akan mengubah warna secara
                    signifikan.

                    \begin{itemize}
                        \item \textbf{Contoh Menyisipkan Bit:}
                              Misalkan:
                              \begin{itemize}
                                  \item Nilai awal komponen warna:
                                        \[
                                            R = 10110011, \quad G = 11001001, \quad B = 11111110
                                        \]
                                  \item Bit pesan yang akan disisipkan: 0, 1, 1.
                              \end{itemize}

                        \item \textbf{Langkah Perubahan:}
                              \begin{itemize}
                                  \item R (Red):
                                        \begin{itemize}
                                            \item LSB sebelum: 1
                                            \item Ganti dengan bit pesan 0.
                                            \item Nilai R menjadi: 10110010.
                                        \end{itemize}
                                  \item G (Green):
                                        \begin{itemize}
                                            \item LSB sebelum: 1
                                            \item Ganti dengan bit pesan 1.
                                            \item Nilai G tetap: 11001001.
                                        \end{itemize}
                                  \item B (Blue):
                                        \begin{itemize}
                                            \item LSB sebelum: 0
                                            \item Ganti dengan bit pesan 1.
                                            \item Nilai B menjadi: 11111111.
                                        \end{itemize}
                              \end{itemize}

                        \item \textbf{Hasil Pixel:}
                              \[
                                  R = 10110010, \quad G = 11001001, \quad B = 11111111.
                              \]
                    \end{itemize}

                    Proses terus diulang untuk setiap pixel, bergerak dari kiri ke kanan, atas ke
                    bawah. Jika semua bit pesan telah disisipkan, pixel sisanya tidak diubah.

                    Encoding berhenti setelah semua bit dari pesan (termasuk terminator
                    \texttt{\textbackslash 000}) berhasil disisipkan. Gambar baru (dengan pesan
                    tersembunyi) disimpan dalam format PNG agar tidak terjadi kompresi lossy.
              \item \textbf{Ilustrasi}

                    Misalkan gambar memiliki pixel sebagai berikut:

                    \begin{quote}
                        \texttt{Pixel (R, G, B) = (10110011, 11001001, 11111110)}
                    \end{quote}

                    Dan pesan yang akan disisipkan adalah huruf "A".

                    \begin{quote}
                        \texttt{"A" dalam ASCII = 65 $\rightarrow$ 01000001 (8 bit).}
                    \end{quote}

                    \textbf{Proses Iterasi:}

                    \begin{table}[H]
                        \centering
                        \begin{tabular}{|c|c|c|c|c|}
                            \hline
                            \textbf{Bit Pesan} & \textbf{Komponen Warna} & \textbf{LSB Lama} & \textbf{LSB Baru} & \textbf{Hasil Akhir} \\ \hline
                            0                  & R (Red)                 & 1                 & 0                 & 10110010             \\ \hline
                            1                  & G (Green)               & 1                 & 1                 & 11001001             \\ \hline
                            0                  & B (Blue)                & 0                 & 0                 & 11111110             \\ \hline
                            0                  & R (Red)                 & 0                 & 0                 & 10110010             \\ \hline
                            0                  & G (Green)               & 1                 & 0                 & 11001000             \\ \hline
                            0                  & B (Blue)                & 0                 & 0                 & 11111110             \\ \hline
                            1                  & R (Red)                 & 0                 & 1                 & 10110011             \\ \hline
                            0                  & G (Green)               & 0                 & 0                 & 11001000             \\ \hline
                        \end{tabular}
                        \caption{Proses Penyisipan Bit Pesan ke dalam Komponen Warna}
                    \end{table}

                    \textbf{Output Gambar:}

                    Gambar baru memiliki nilai komponen warna yang dimodifikasi pada LSB, tetapi
                    perubahan warna tidak terlihat oleh mata manusia. Gambar ini berisi pesan yang
                    telah disisipkan secara tersembunyi.

          \end{enumerate}

    \item Generating Key Trace

          Proses pembuatan kunci yang akan menentukan di mana dan bagaimana data akan
          disembunyikan dalam gambar. Sistem menggunakan cara khusus untuk membuat pola
          acak yang memastikan data tersebar merata dalam gambar. Cara ini membuat sistem
          lebih aman karena menambah lapisan pengamanan, sekaligus memudahkan pengambilan
          data kembali tanpa merusak gambar. Jika terjadi masalah atau ada yang mencoba
          mengambil data secara ilegal, sistem ini juga membantu mengelola dan memulihkan
          data dengan lebih baik.

    \item Agreed Image

          Gambar yang telah ditentukan bersama antara pengirim dan penerima sebagai media
          penyimpanan jejak kunci steganografi. Pemilihan gambar ini dilakukan dengan
          mempertimbangkan berbagai karakteristik teknis seperti kompleksitas tekstur,
          distribusi warna, dan noise level yang optimal untuk mendukung proses
          penyembunyian informasi tanpa menimbulkan kecurigaan. Gambar yang dipilih juga
          harus memiliki kapasitas yang memadai untuk menampung jejak kunci sambil tetap
          mempertahankan kualitas visual yang baik setelah proses penyisipan.

    \item Embedding Key Trace into Agreed Image

          Tahap kritis dimana key trace disisipkan ke dalam gambar yang telah disepakati
          sebelumnya menggunakan teknik steganografi. Proses ini memerlukan presisi untuk
          memastikan integritas data dan imperceptibility optimal. Penyisipan dilakukan
          dengan mempertimbangkan karakteristik gambar dan distribusi nilai pixel untuk
          mengoptimalkan keseimbangan antara kapasitas penyimpanan dan kualitas visual.
          Teknik steganografi yang digunakan juga mempertimbangkan aspek keamanan untuk
          mencegah deteksi dan ekstraksi unauthorized.

    \item Agreed Stego

          Hasil akhir dari proses penyisipan jejak kunci ke dalam gambar yang telah
          disepakati bersama. Gambar ini memiliki karakteristik visual yang identik
          dengan gambar asli untuk menghindari kecurigaan, namun di dalamnya telah
          tertanam informasi rahasia dalam bentuk jejak kunci yang akan digunakan untuk
          proses ekstraksi data. Kualitas gambar tetap terjaga meskipun telah melalui
          proses penyisipan, dengan perubahan nilai pixel yang minimal dan tidak
          terdeteksi oleh mata manusia. Gambar ini berperan penting sebagai pembawa
          informasi kontrol yang diperlukan untuk proses steganografi selanjutnya.

    \item Embedding secret bits into Cover Image

          Embedding secret bits into cover image merupakan proses penyisipan bit-bit data
          rahasia ke dalam gambar cover dilakukan dengan memanfaatkan algoritma
          steganografi yang canggih dan teroptimasi. Proses ini melibatkan analisis
          karakteristik gambar cover untuk menentukan lokasi optimal penyisipan data,
          kemudian menggunakan teknik transformasi yang presisi untuk menyisipkan bit-bit
          informasi rahasia. Algoritma yang digunakan dirancang khusus untuk
          memaksimalkan kapasitas penyimpanan data sambil tetap mempertahankan kualitas
          visual gambar, dengan fokus khusus pada minimalisasi distorsi yang dapat
          terdeteksi baik secara visual maupun statistik. Proses ini juga
          mempertimbangkan aspek keseimbangan antara efisiensi penyisipan dan ketahanan
          terhadap berbagai teknik steganalisis.

    \item Cover Image

          Cover image merupakan gambar yang berfungsi sebagai media utama untuk
          penyembunyian data rahasia, dipilih dengan pertimbangan khusus berdasarkan
          kompleksitas tekstur, distribusi warna, dan karakteristik statistik yang
          optimal untuk mendukung proses penyisipan informasi. Pemilihan cover image yang
          tepat sangat kritis karena akan mempengaruhi kapasitas penyimpanan data,
          ketahanan terhadap deteksi, dan kualitas visual hasil akhir steganografi.
          Gambar dengan area tekstur yang kompleks dan variasi nilai pixel yang tinggi
          umumnya lebih ideal karena dapat menyembunyikan perubahan yang diakibatkan oleh
          proses penyisipan data dengan lebih efektif.

    \item Generate random pixels positions

          Algoritma untuk menghasilkan posisi piksel secara acak namun tetap
          deterministik berdasarkan seed value yang ditentukan. Proses ini menggunakan
          pembangkit bilangan pseudo-random yang telah terverifikasi untuk menghasilkan
          sekuens posisi piksel yang terdistribusi merata di seluruh gambar. Pendekatan
          deterministik ini penting untuk memastikan konsistensi dalam proses penyisipan
          dan ekstraksi data, sekaligus meningkatkan keamanan dengan menciptakan pola
          penyebaran data yang sulit diprediksi oleh pihak yang tidak berwenang.

    \item Generating ECC Key Pairs

          Proses pembuatan sepasang kunci kriptografi menggunakan algoritma Elliptic
          Curve Cryptography (ECC), yang menyediakan tingkat keamanan yang sangat tinggi
          dengan ukuran kunci yang relatif kecil dibandingkan dengan sistem kriptografi
          konvensional. Pemilihan parameter kurva eliptik dan proses pembangkitan kunci
          dilakukan dengan mempertimbangkan aspek keamanan dan efisiensi komputasi.

    \item Embedding Public Key into Stego Image

          Proses penyisipan kunci publik ke dalam gambar stego dilakukan dengan
          menggunakan teknik steganografi yang telah dioptimasi untuk menjaga kualitas
          visual gambar. Penyisipan ini memungkinkan verifikasi dan dekripsi yang aman
          oleh penerima yang berwenang, sekaligus menjamin integritas dan otentisitas
          data yang tersembunyi. Proses ini dirancang untuk meminimalkan dampak visual
          dari penyisipan kunci publik.

    \item Private Key

          Kunci pribadi yang dihasilkan dalam proses ECC disimpan secara terpisah dengan
          menerapkan protokol keamanan berlapis dan enkripsi tambahan. Kunci ini memiliki
          tingkat keamanan yang sangat tinggi karena perannya yang kritis dalam proses
          dekripsi dan verifikasi data tersembunyi. Penyimpanan dan pengelolaan kunci
          pribadi mengikuti standar keamanan industri untuk mencegah akses tidak sah.

    \item Stego Image

          Gambar hasil akhir setelah proses penyisipan data rahasia, yang dirancang untuk
          mempertahankan karakteristik visual yang identik dengan gambar cover asli.
          Meskipun mengandung informasi tersembunyi, gambar stego tidak menunjukkan
          perbedaan yang dapat terdeteksi secara visual maupun statistik sederhana.
          Kualitas gambar dijaga melalui optimasi algoritma penyisipan yang
          mempertimbangkan karakteristik persepsi visual manusia.

    \item Final Stego Image

          Versi terakhir dari gambar stego yang telah dilengkapi dengan kunci publik dan
          semua metadata yang diperlukan untuk proses ekstraksi dan verifikasi yang aman.
          Gambar ini menyediakan mekanisme pemulihan data yang komprehensif namun tetap
          mempertahankan aspek keamanan dan kerahasiaan informasi yang tersembunyi.
          Struktur data tambahan diintegrasikan dengan cara yang tidak mengganggu
          kualitas visual keseluruhan.
    \item \textbf{Decoding Process}

          Proses decoding bertujuan untuk mengekstrak pesan tersembunyi yang telah
          disisipkan ke dalam gambar menggunakan metode \textbf{Least Significant Bit
              (LSB)}. Berikut adalah langkah-langkah decoding:

          \begin{enumerate}
              \item \textbf{Persiapan Data}
                    \begin{itemize}
                        \item \textbf{Input:} Gambar yang telah di-encode dengan pesan tersembunyi.
                        \item \textbf{Tujuan:} Mengekstrak bit dari komponen warna setiap piksel dan menyusun kembali menjadi karakter pesan.
                    \end{itemize}

              \item \textbf{Iterasi Gambar Pixel-per-Pixel} \\
                    Gambar diakses pixel-per-pixel (dari kiri ke kanan, atas ke bawah), lalu dilakukan pembacaan bit dari komponen warna:
                    \begin{itemize}
                        \item Komponen warna \textbf{R (Red)}, \textbf{G (Green)}, dan \textbf{B (Blue)}
                              diekstrak satu per satu.
                        \item Dari setiap komponen warna, \textbf{Least Significant Bit (LSB)} dibaca karena
                              LSB digunakan untuk menyisipkan bit pesan.
                    \end{itemize}

                    \noindent
                    \textbf{Contoh:}
                    \[
                        \begin{array}{rl}
                            R & = 10110010 \\
                            G & = 11001001 \\
                            B & = 11111111
                        \end{array}
                    \]
                    % R = 10110011, \quad G = 11001001, \quad B = 11111110

                    \noindent
                    LSB dari setiap komponen warna:
                    \begin{itemize}
                        \item LSB $R = 0$
                        \item LSB $G = 1$
                        \item LSB $B = 1$
                    \end{itemize}

                    \noindent
                    Bit yang diperoleh dari piksel tersebut adalah: $0, 1, 1$.

              \item \textbf{Penyusunan Bit Menjadi Byte} \\
                    Setiap 8 bit yang diperoleh dari LSB disusun kembali menjadi 1 byte.
                    \begin{itemize}
                        \item LSB pertama masuk ke bit paling kiri.
                        \item LSB berikutnya diikuti hingga terbentuk 8 bit (1 byte).
                    \end{itemize}

                    \noindent
                    \textbf{Contoh:}
                    \begin{center}
                        LSB yang dibaca: $0, 1, 0, 0, 1, 0, 0, 0$ \\
                        Hasil: $01001000$, yang dalam ASCII adalah huruf \textbf{H}.
                    \end{center}

                    Proses pembacaan bit dari LSB dilanjutkan hingga menemukan \textbf{null
                        terminator} ($\backslash 000$), yaitu byte dengan nilai \texttt{0x00}
                    ($00000000$ dalam biner).

                    \noindent
                    \textbf{Contoh:}
                    \[
                        \texttt{H → e → l → l → o → , → W → o → r → l → d → !}
                    \]

                    Diikuti oleh null byte $\backslash 000$, proses decoding berhenti.

                    Byte yang telah dibaca dikonversi kembali menjadi karakter ASCII dan disusun
                    menjadi string pesan yang telah disembunyikan.

          \end{enumerate}

          \noindent
          \textbf{Ringkasan Proses Decoding:}
          \begin{itemize}
              \item Baca LSB dari komponen warna $R$, $G$, dan $B$ secara berurutan.
              \item Susun LSB menjadi 8 bit untuk membentuk 1 byte.
              \item Ulangi proses hingga menemukan \textbf{null terminator} ($\backslash 000$).
              \item Konversi byte yang terkumpul menjadi string pesan.
              \item (Opsional) Validasi pesan dengan menghasilkan \textbf{key trace} menggunakan \textbf{SHA-256}.
          \end{itemize}
\end{enumerate}

\subsection{Proses Implementasi}

Algoritma yang dirancang akan diimplementasikan menggunakan Python,
memanfaatkan library seperti OpenCV untuk pemrosesan gambar dan NumPy untuk
operasi numerik. Implementasi ini dapat mendukung gambar RGB maupun grayscale
untuk memungkinkan analisis komparatif.

\subsubsection{Persiapan Environment Development}

Mulai dengan instalasi Python sebagai bahasa pemrograman utama. Instal library
OpenCV yang akan digunakan untuk membaca, menulis, dan memanipulasi gambar
digital. Tambahkan NumPy untuk mendukung operasi matematika dan array yang
diperlukan dalam pemrosesan gambar.

\subsubsection{Pengembangan Modul Utama}

\begin{enumerate}
    \item Implementasi fungsi pembacaan gambar:
          \begin{enumerate}
              \item Pembacaan gambar format RGB menggunakan OpenCV
              \item Pembacaan gambar format grayscale menggunakan OpenCV
              \item Validasi format input gambar
          \end{enumerate}

    \item Implementasi fungsi pemrosesan pixel:
          \begin{enumerate}
              \item Modifikasi nilai pixel untuk penyisipan data
              \item Pengecekan kapasitas penyisipan
              \item Validasi integritas data
          \end{enumerate}

    \item Implementasi fungsi ekstraksi data:
          \begin{enumerate}
              \item Pembacaan nilai pixel termodifikasi
              \item Ekstraksi data tersembunyi
              \item Rekonstruksi pesan asli
          \end{enumerate}

    \item Implementasi fungsi evaluasi kualitas:
          \begin{enumerate}
              \item Perhitungan PSNR (Peak Signal-to-Noise Ratio)
              \item Perhitungan MSE (Mean Square Error)
              \item Analisis imperceptibility hasil steganografi
          \end{enumerate}
\end{enumerate}

\subsubsection{Implementasi Algoritma Steganografi}

Dalam implementasi algoritma steganografi, pengembangan sistem dimulai dengan
implementasi mekanisme penyisipan ganda yang komprehensif. Proses ini mencakup
identifikasi area potensial untuk penyisipan data, pengembangan metode
penyisipan bertingkat, serta pengelolaan metadata untuk multiple embedding.
Koordinasi antar layer penyisipan menjadi aspek krusial untuk memastikan
integritas data yang disisipkan.

Sistem yang dikembangkan dirancang untuk mendukung berbagai format gambar,
dengan fokus utama pada penanganan gambar RGB (24-bit) dan grayscale (8-bit).
Adaptasi algoritma dilakukan untuk mengakomodasi kedua format tersebut,
disertai dengan optimasi performa yang memungkinkan sistem bekerja secara
efisien pada berbagai jenis gambar carrier.

Optimasi kapasitas penyisipan menjadi salah satu prioritas utama, dengan target
minimal mencapai 50\% dari ukuran carrier. Hal ini dicapai melalui implementasi
teknik bit-plane slicing yang efisien, didukung oleh mekanisme kompresi data
payload, serta pemilihan pixel secara adaptif untuk memaksimalkan kapasitas
penyimpanan tanpa mengorbankan kualitas visual.

Preservasi kualitas visual menjadi aspek fundamental dalam pengembangan, dengan
target PSNR yang ditetapkan di atas 40 dB. Sistem mengimplementasikan berbagai
strategi untuk meminimalisasi distorsi visual, termasuk teknik penyebaran
perubahan pixel dan adaptasi terhadap karakteristik spesifik dari gambar
carrier.

Implementasi dilengkapi dengan mekanisme validasi yang komprehensif, mencakup
verifikasi integritas data, pengukuran kapasitas efektif, evaluasi kualitas
visual, serta pengujian robustness. Serangkaian pengujian ini memastikan bahwa
sistem yang dikembangkan memenuhi seluruh requirement yang telah ditetapkan,
baik dari segi kapasitas, kualitas, maupun keamanan data.

\subsubsection{Pengembangan Sistem Pengujian}

Buat modul pengujian untuk:

- Mengukur kapasitas maksimum penyisipan data
- Mengevaluasi kualitas visual menggunakan PSNR dan SSIM
- Menguji ketahanan terhadap teknik steganalisis
- Membandingkan dengan metode LSB konvensional

\subsubsection{Analisis dan Evaluasi}

Lakukan analisis komprehensif meliputi:

- Pengujian statistik menggunakan uji-t dan ANOVA
- Evaluasi visual subjektif
- Pengukuran metrik kinerja (kapasitas, PSNR, SSIM)
- Evaluasi efisiensi komputasi

\subsubsection{Dokumentasi}

Menyiapkan dokumentasi lengkap yang mencakup:

\begin{enumerate}
    \item Source code dengan komentar yang jelas
    \item Manual penggunaan sistem
    \item Hasil pengujian dan analisis
    \item Rekomendasi untuk pengembangan lebih lanjut
\end{enumerate}

\subsection{Desain Eksperimen dan Pengumpulan Data}

Untuk mengevaluasi kinerja metode yang diusulkan dibandingkan dengan metode
lain, sebuah eksperimen dilakukan berdasarkan empat ukuran kuantitatif utama.
Gambar \ref{fig:experiment} mengilustrasikan desain eksperimen yang disarankan
untuk metode yang diusulkan.

% flowchart LR
%     CDS["CDS Experiment"] --> SEC["SECURITY"]
%     CDS --> IMP["IMPERCEPTIBILITY"]
%     CDS --> ROB["ROBUSTNESS"]
%     CDS --> CAP["CAPACITY"]

%     SEC --> STEG["STEGANALYSIS"]
%     SEC --> SECP["SECURITY PROTOCOL"]

%     ROB --> NOISE["NOISE"]
%     ROB --> JPG["JPG COMPRESSION"]
%     ROB --> SCR["SCRATCHING"]
%     ROB --> CROP["CROPPING"]

%     CAP --> SIS["SECRET IMAGE SIZES"]
%     CAP --> VIH["VARIATION IN HISTOGRAMS"]

%     STEG --> TRIP["TRIPLES ANALYSIS"]

%     SCR  --> SINGLE["SINGLE"]
%     SCR  --> MULT["MULTIPLE"]
%     CROP --> SIZE["SIZE"]
%     CROP --> REG["REGION"]

%     NOISE --> GAUSS["GAUSSIAN"]
%     NOISE --> SP["SALT & PEPPER"]
%     NOISE --> SPEC["SPECKLE"]

%     VIH --> CI["COVER IMAGES"]
%     VIH --> SI["SECRET IMAGES"]

\begin{figure}[htb]
    \centering
    \includegraphics[width=0.8\textwidth]{gambar/experiment.png}
    \caption{Design experiment}
    \label{fig:experiment}
\end{figure}

\subsubsection{Imperceptibility}

% apa yang mau divariasikan, untuk MSE & PSNR masuk ke analisis
% Jelaskan single multi dll

Skema steganografi tidak akan terlihat jika mata manusia tidak dapat membedakan
antara sampul dan gambar stego. Yang menjadi perhatian dalam bagian ini adalah
kualitas visual gambar stego yang dihasilkan. Kelompok gambar rahasia dengan
ukuran gambar yang sama dan distribusi nada yang beragam terbentuk. Setiap
gambar rahasia dalam kelompok tertentu disematkan di sampul menggunakan metode
yang diusulkan dan metode A. M. Molaei.

Ketidaksadaran kemudian diukur dengan Mean Square Error (MSE) dan Peak to
Signal Noise Ratio (PSNR). PSNR yang lebih tinggi menunjukkan bahwa gambar
stego lebih mirip dengan gambar asli, yang berarti kualitas visual lebih baik.

\subsubsection{Robbustnes}

Pada bagian ini menguji kemampuan metode yang diusulkan untuk mengatasi
berbagai serangan. Hal ini memungkinkan penerima untuk mengambil kembali pesan
rahasia yang telah dihancurkan jika terjadi penghancuran citra stego secara
sengaja atau tidak sengaja. Ukuran kuantitatif yang akan ditentukan adalah PSNR
dari citra rahasia yang dipulihkan.

Setelah setiap skenario penyematan dilakukan seperti yang dijelaskan, gambar
stego yang dihasilkan diekspos di bawah serangan berikut

% NOTES
% jelaskan metode komparasinya
% kasih warna lain untuk kontribusi
% sebagai berikut = tidak formal

\begin{enumerate}
    \item Noise Attack: Gambar stego diserang dengan beberapa metode di bawah dengan
          intensitas yang berbeda.
          \begin{enumerate}
              \item \textbf{Gaussian Noise}: Nama ini diambil dari Carl Friedrich Gauss
                    \cite{selamistudy}. Gaussian Noise merupakan noise statistik yang memiliki
                    fungsi kerapatan probabilitas (PDF) yang sama dengan distribusi normal, yang
                    juga dikenal sebagai distribusi Gaussian. Dengan kata lain, nilai yang dapat
                    diambil oleh noise tersebut berdistribusi Gaussian. Untuk Gaussian Noise, dua
                    variasi varians dipertimbangkan dalam eksperimen, yaitu Gaussian noise: mean 0
                    variance 0,01 dan Gaussian noise: mean 0 variance 0,1.
              \item \textbf{Salt and Pepper Noise}: Ini adalah bentuk noise yang juga terlihat pada
                    gambar. Ini juga dikenal sebagai noise impuls. Noise ini dapat disebabkan oleh
                    gangguan tajam dan tiba-tiba pada sinyal gambar. Noise ini muncul sebagai
                    piksel putih dan hitam yang jarang muncul \cite{kaisar2008salt}. Noise Salt and Pepper hadir
                    dalam berbagai bentuk dengan kepadatan yang berbeda. Dua variasi kepadatan
                    dipertimbangkan dalam percobaan yaitu noise Salt and Pepper: kepadatan 0,05 dan
                    noise Salt and Pepper: kepadatan 0,5.
              \item \textbf{Speckle Noise}: Speckle noise adalah noise yang muncul akibat pengaruh kondisi
                    lingkungan terhadap sensor pencitraan selama akuisisi citra. Speckle noise
                    paling banyak terdeteksi pada citra medis, citra Radar aktif, dan citra
                    Synthetic Aperture Radar (SAR) \cite{mansourpour2006effects}. Dua variasi varians dipertimbangkan dalam
                    eksperimen yaitu Speckle noise: varians 0,04 dan Speckle noise: varians 0,4.
          \end{enumerate}

          Setelah stego terkena serangan Noise ini, proses ekstraksi mengambil gambar
          rahasia dan melakukan koreksi kesalahan. PSNR dari gambar rahasia yang diambil
          ditentukan dan ditabulasi. PSNR dari semua gambar rahasia yang diambil
          ditentukan dan dibandingkan dengan Metode A. M. Molaei.

    \item Cropping Attack: Citra stego dengan rentang nilai PSNR yang sama dikelompokkan
          dan diekspos ke dalam serangan pemotongan. Serangan pemotongan dilakukan
          berdasarkan rasio pemotongan dan wilayah pemotongan yang berbeda. Setiap citra
          stego dalam kelompok tertentu diekspos ke dalam rasio pemotongan dan wilayah
          pemotongan yang sama dengan citra stego lain dalam kelompok yang sama. Citra
          stego akan menjalani proses ekstraksi untuk mengambil citra rahasia dengan
          melakukan koreksi kesalahan. Untuk setiap set atau kelompok, nilai PSNR dari
          citra rahasia yang diambil ditentukan dan ditabulasi serta dibandingkan dengan
          Metode A. M. Molaei.
    \item Scratching Attack: Serangan goresan muncul dalam dua bentuk berbeda yaitu
          perbedaan panjang, lebar sama dan perbedaan lebar, panjang sama. Setelah setiap
          bentuk serangan goresan pada gambar stego dilakukan, koreksi kesalahan
          digunakan untuk mengambil gambar rahasia. PSNR dari gambar rahasia yang diambil
          ditentukan, ditabulasi dan dibandingkan dengan metode sebelumnya.

          Scratching single merupakan Scratching yang dilakukan dengan cara menggores
          satu garis pada gambar stego. Scratching multiple merupakan Scratching yang
          dilakukan dengan cara menggores beberapa garis pada gambar stego.
    \item JPEG Compression Attack: Singkatan dari Joint Photographic Experts Group. JPEG
          adalah metode kompresi lossy yang umum digunakan untuk gambar digital, terutama
          untuk gambar yang dihasilkan oleh fotografi digital. JPEG biasanya mencapai
          kompresi 10:1 dengan sedikit kehilangan kualitas gambar yang
          terlihat.\cite{ch2015medical} Kompresi JPEG digunakan dalam sejumlah format
          berkas gambar. JPEG adalah format gambar yang paling umum digunakan oleh kamera
          digital dan format yang paling umum untuk menyimpan dan mengirimkan gambar
          fotografi di Web.

          Untuk melakukan percobaan, semua gambar stego yang disematkan dengan ukuran
          berbeda dari pesan rahasia yang sama diekspos ke serangan JPG. Setelah
          mengekspos stego ke serangan kompresi JPEG, ekstraksi gambar rahasia dilakukan
          dengan menerapkan koreksi kesalahan untuk memulihkan gambar rahasia. PSNR dari
          gambar rahasia yang dipulihkan ditentukan, ditabulasi, dan dibandingkan dengan
          Metode A. M. Molaei.
\end{enumerate}

\subsubsection{Kapasitas}

Pada bagian ini, kapasitas penyisipan dari metode yang diusulkan dan metode
yang digunakan diperiksa. Rasio jumlah bit rahasia yang disematkan dalam gambar
Cover terhadap jumlah piksel gambar Cover dievaluasi.

Hasil untuk setiap kelompok ditabulasi dan nilai PSNR ditentukan dan
dibandingkan dengan Metode sebelumnya.

\subsection{Metode Analisis dan Evaluasi}

Penelitian ini akan merancang serangkaian eksperimen untuk mengevaluasi kinerja
metode yang diusulkan:

\subsubsection{Imperceptibility}

% apa yang mau divariasikan, untuk MSE & PSNR masuk ke analisis
% Jelaskan single multi dll

Mean Square Error (MSE) adalah metrik yang digunakan untuk mengukur perbedaan
antara dua gambar. MSE dihitung dengan rumus berikut:

\begin{equation}
    \text{MSE} = \frac{1}{MN} \sum_{i=1}^{M} \sum_{j=1}^{N} [I(i,j) - K(i,j)]^2
\end{equation}

Di mana:
\begin{itemize}
    \item $M$ dan $N$ adalah dimensi gambar.
    \item $I(i,j)$ adalah nilai piksel pada posisi $(i,j)$ dari gambar asli.
    \item $K(i,j)$ adalah nilai piksel pada posisi $(i,j)$ dari gambar stego.
\end{itemize}

MSE yang lebih rendah menunjukkan bahwa gambar stego lebih mirip dengan gambar
asli, yang berarti kualitas visual lebih baik.

Peak Signal-to-Noise Ratio (PSNR) adalah metrik yang digunakan untuk mengukur
kualitas gambar stego dibandingkan dengan gambar asli. PSNR dihitung dengan
rumus berikut:

\begin{equation}
    \text{PSNR} = 10 \cdot \log_{10} \left( \frac{MAX_I^2}{\text{MSE}} \right)
\end{equation}

Di mana:
\begin{itemize}
    \item $MAX_I$ adalah nilai maksimum intensitas gambar (misalnya, 255 untuk gambar 8-bit).
    \item $\text{MSE}$ adalah Mean Square Error antara gambar asli dan gambar stego.
\end{itemize}

PSNR yang lebih tinggi menunjukkan bahwa gambar stego lebih mirip dengan gambar
asli, yang berarti kualitas visual lebih baik.

\subsubsection{Robbustnes}

Ukuran kuantitatif yang akan ditentukan adalah PSNR dari citra rahasia yang
dipulihkan. Hal ini ditentukan menggunakan persamaan \ref{eq:rm} dan
\ref{eq:rm2}:

Kode Reed Muller RM (1, 4) telah diterapkan dalam metode ini karena kemampuan
koreksi kesalahannya yang lebih tinggi. Hal ini ditunjukkan sebagai berikut:

Karena r = 1 dan m = 4, maka ukuran pesan k dihitung sebagai berikut:
\begin{equation}
    k = \sum_{i=0}^{r} \binom{m}{i} = \binom{4}{0} + \binom{4}{1} = 5 \text{ bits}
    \label{eq:rm}
\end{equation}

Dan kode berukuran N dihitung sebagai berikut:

\begin{equation}
    N = 2^m = 16 \text{ bits}
    \label{eq:rm2}
\end{equation}

\begin{equation}
    ECC = \frac{\text{number of correctable bits}}{\text{total bits}} = \frac{3}{16} = 18.75\%
\end{equation}

Setelah setiap skenario penyematan dilakukan seperti yang dijelaskan, gambar
stego yang dihasilkan diekspos di bawah serangan berikut

\begin{enumerate}
    \item Noise Attack
    \item Cropping Attack
    \item Scratching Attack
    \item JPEG Compression Attack
\end{enumerate}

\subsubsection{Kapasitas}

Kapasitas penyisipan, EC kemudian ditentukan oleh persamaan berikut:

\begin{equation}
    EC = \frac{|S|}{H \cdot W}
\end{equation}

Di mana —S— adalah jumlah bit rahasia yang disematkan, H dan W masing-masing
adalah tinggi dan panjang gambar. Muatan yang dihasilkan dinyatakan sebagai
persentase. Untuk mengevaluasi kinerja metode yang diusulkan dalam hal
kapasitas, dua skenario utama berikut dipertimbangkan:

\begin{enumerate}
    \item Variasi Nilai Piksel: Gambar sampul dan gambar rahasia dianalisa berdasarkan
          histogram gambarnya.
          \begin{itemize}
              \item Digunakan 40 gambar sampul skala berwarna berukuran 512x512 piksel.
              \item Digunakan 40 gambar rahasia skala berwarna dengan berbagai ukuran.
              \item Setiap gambar rahasia dengan distribusi tonal unik disematkan pada setiap
                    gambar sampul dengan tonal yang unik pula.
          \end{itemize}
    \item Ukuran Secret Image: Untuk setiap gambar rahasia, dilakukan 33 variasi ukuran
          yang berbeda.
          \begin{itemize}
              \item Setiap variasi ukuran disematkan pada gambar sampul yang berbeda.
              \item Kapasitas yang dihasilkan dari setiap penyematan dicatat.
          \end{itemize}
\end{enumerate}

Hasil untuk setiap kelompok ditabulasi dan nilai PSNR ditentukan dan
dibandingkan dengan Metode sebelumnya.

% Research method should contain brief description on fundamental knowledge related to the addressed problem. You should highlight that the concept is appropriate for your problem. Write down the mapping between the concept used and the problem solved.

% Also, it defines the logic steps - What to do and how to solve the problem and achieve proposed objectives? Which research methods (e.g. survey, modeling, case study ...) will be used? 

% More specifically, the section comprises the following components:
% \begin{itemize}
% \item[1.] Requirement identification
% \item[2.] Design process (including general system architecture)
% \item[3.] Implementation process
% \item[4.] Experiment design and plan (including data collection process)
% \item[5.] Analysis/Evaluation method which will be used for analyzing the experiment result
% \end{itemize}

% To complete this section, see the detail in the assignment form of assessment-4 and assessment-5.

\section{Work Plan and Time Schedule}
Write a work plan along with the schedule for completion. The following is the
example. You may adjust the activities and time schedule according to the
problem. \newline

\begin{table}[h!]
    \caption{Activity Schedule (\color{red}example) \label{tab:schedule}}
    \noindent\begin{tabularx}{\linewidth}{|>{\bfseries}l|l|*{11}{>{\centering\arraybackslash}X|}>{\centering\arraybackslash}X<{\bigstrut}|}
        \hline
        \multicolumn{2}{|l|}{}                   & \multicolumn{12}{c|}{\bfseries SEMESTER\bigstrut}                                                                                                                                                                                           \\
        \cline{3-14}
        \multicolumn{2}{|c|}{\bfseries Activity} & \multicolumn{3}{c|}{\bfseries 1}                  & \multicolumn{3}{c|}{\bfseries 2} & \multicolumn{3}{c|}{\bfseries 3} & \multicolumn{3}{c|}{\bfseries 4\bigstrut}                                                                         \\
        \hline

        1                                        & Literature study                                  & \blue                            &                                  &                                           &       &       &       &       &       &       &       &       &       \\
        \hline
        2                                        & Problem identification                            & \blue                            & \blue                            &                                           &       &       &       &       &       &       &       &       &       \\
        \hline
        3                                        & Contribution formulation                          & \blue                            & \blue                            &                                           &       &       &       &       &       &       &       &       &       \\
        \hline
        4                                        & Hypothesis formulation                            & \blue                            & \blue                            &                                           &       &       &       &       &       &       &       &       &       \\
        \hline
        5                                        & Proposal                                          &                                  &                                  & \blue                                     &       &       &       &       &       &       &       &       &       \\
        \hline
        6                                        & Encoding Process                                  &                                  &                                  &                                           & \blue & \blue &       &       &       &       &       &       &       \\
        \hline
        7                                        & Decoding Process                                  &                                  &                                  &                                           &       & \blue & \blue &       &       &       &       &       &       \\
        \hline
        8                                        & Implementation process                            &                                  &                                  &                                           &       &       &       & \blue & \blue & \blue &       &       &       \\
        \hline
        9                                        & Experiment design                                 &                                  &                                  &                                           &       &       &       &       &       & \blue &       &       &       \\
        \hline
        10                                       & Evaluation and analysis                           &                                  &                                  &                                           &       &       &       &       &       & \blue & \blue & \blue &       \\
        \hline
        11                                       & Thesis draft                                      &                                  &                                  &                                           &       &       &       &       &       &       & \blue & \blue & \blue \\
        \hline

    \end{tabularx}
\end{table}

%\section{Summary}
\supervisorcomments

\printbibliography

\end{document}
