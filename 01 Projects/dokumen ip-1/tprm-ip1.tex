\documentclass{ittelkom}
\usepackage{graphicx}
\usepackage{multirow}
\usepackage{tabularx, ragged2e, lmodern, bigstrut}
\usepackage[sgvnames, table]{xcolor}
\usepackage{cite} % Tambahkan paket cite untuk penanganan sitasi yang lebih baik

\newcommand{\blue}{\cellcolor{blue!75}}

\begin{document}

\laporancapstonecover{April, 18}{2025}{RGB Image Steganography Using Multiple Embedding}{Dimas Anwar Aziz}{203022410012}{Prof. Ari Moesriami Barmawi, Ph.D.}{-}{dimasanwaraziz@student.telkomuniversity.ac.id}
% \laporancapstonecover{April, 18}{2025}{Multiple Embedding steganography for RGB Image}{Dimas Anwar Aziz}{203022410012}{Prof. Ari Moesriami Barmawi, Ph.D.}{-}{dimasanwaraziz@student.telkomuniversity.ac.id}

\newpage
\normalsize
\tableofcontents

\newpage
\section*{Progress Summary: (max 500 words)}
\label{summary}
\justify{
    Penelitian mengenai steganografi Multiple Embedding untuk gambar RGB telah mencapai kemajuan. Tahapan studi literatur, identifikasi masalah, formulasi kontribusi, dan penyusunan proposal telah diselesaikan. Saat ini, fokus penelitian berada pada tahap desain dan implementasi metode yang diusulkan. Berdasarkan studi awal dan tinjauan literatur, teridentifikasi bahwa metode sebelumnya yang menggunakan pendekatan berbasis kode seperti Reed-Muller memiliki keterbatasan, terutama dalam hal kapasitas pada gambar grayscale \cite{kingsley2020improving}. Oleh karena itu, penelitian ini mengusulkan penggunaan Polar Codes yang dikombinasikan dengan teknik multiple embedding untuk gambar RGB. Tujuannya adalah meningkatkan kapasitas penyembunyian data secara signifikan (target minimal 50\% lebih tinggi dari metode Kingsley et al. \cite{kingsley2020improving}) sambil mempertahankan kualitas visual gambar stego (PSNR $>$ 40 dB).

    Proses desain metode, termasuk alur kerja encoding, pembangkitan jejak kunci,
    penyisipan kunci dan data rahasia menggunakan Polar Codes, serta skema
    pembagian kunci (secret sharing) telah dirancang. Tahap implementasi awal
    sedang berjalan, mencakup pengembangan fungsi-fungsi inti dalam Python
    menggunakan library seperti OpenCV dan NumPy. Tantangan utama saat ini adalah
    mengoptimalkan algoritma Polar Codes untuk penyisipan pada ketiga channel warna
    RGB secara efisien dan memastikan integrasi yang mulus dengan mekanisme
    multiple embedding. Tahap selanjutnya adalah pengujian awal imperceptibility
    dan kapasitas, diikuti dengan pengujian robustness terhadap berbagai serangan.
    Kesulitan yang dihadapi terutama berkaitan dengan kompleksitas matematis Polar
    Codes dan penyesuaiannya untuk domain steganografi gambar berwarna. Komentar
    dari pembimbing terkait metode penelitian akan terus diintegrasikan dalam
    pengembangan. Diharapkan tahap implementasi dan pengujian awal dapat
    diselesaikan sesuai jadwal. }

\section{Background} \label{background}
\justify{
    % Keamanan data digital menjadi sangat penting di era digital saat ini untuk melindungi informasi sensitif \cite{pujianto2021uji, siaulhak2023sistem}. Kriptografi dan steganografi adalah dua solusi utama untuk tujuan ini. Steganografi menyembunyikan keberadaan pesan rahasia di dalam media lain, seperti gambar, tanpa terdeteksi \cite{pujianto2021uji}. Umumnya, pesan dienkripsi terlebih dahulu sebelum disembunyikan \cite{soetarmono2012studi}. Gambar digital populer digunakan sebagai media penyembunyian karena ketersediaannya yang luas, kapasitas penyembunyian yang inheren, dan kemudahan manipulasi piksel. Namun, tantangan utamanya adalah menyeimbangkan kapasitas, kualitas gambar, dan ketahanan terhadap deteksi \cite{fikri2022optimasi}. Penelitian ini berfokus pada peningkatan kapasitas steganografi pada gambar berwarna (RGB) yang memiliki tiga channel, melampaui keterbatasan metode sebelumnya yang seringkali fokus pada gambar grayscale. Dengan mengoptimalkan ketiga channel warna, diharapkan kapasitas dapat ditingkatkan signifikan sambil menjaga kualitas visual dan ketahanan.

    Dalam era digital yang semakin berkembang, keamanan data digital menjadi
    prioritas utama. Informasi pribadi, data bisnis, dan data kritikal lainnya
    harus dilindungi dari ancaman kebocoran atau akses yang tidak sah. Dalam
    menghadapi tantangan ini, kriptografi dan steganografi muncul sebagai solusi
    penting untuk meningkatkan keamanan data digital.

    Steganografi merupakan teknik menyembunyikan pesan di dalam suatu media
    penyisipan pesan atau cover image, sehingga keberadaan pesan rahasia yang
    disisipkan tidak dapat dilihat secara langsung \cite{pujianto2021uji}.
    Steganografi memungkinkan pertukaran pesan rahasia melalui penyembunyian
    informasi pada berbagai media digital seperti gambar, video, dan audio, tanpa
    menimbulkan kecurigaan \cite{siaulhak2023sistem}. Steganografi dapat dipandang
    sebagai kelanjutan kriptografi dan dalam prakteknya pesan rahasia dienkripsi
    terlebih dahulu, kemudian cipherteks disembunyikan di dalam media lain sehingga
    pihak ketiga tidak menyadari keberadaannya. Pesan rahasia yang disembunyikan
    dapat diekstraksi kembali persis sama seperti aslinya
    \cite{soetarmono2012studi}.

    Di antara berbagai media digital yang dapat digunakan untuk steganografi,
    gambar menjadi pilihan yang populer karena beberapa alasan. Pertama, gambar
    digital tersedia secara luas dan pertukaran gambar di internet adalah hal yang
    umum, sehingga tidak menimbulkan kecurigaan. Kedua, gambar memiliki kapasitas
    untuk menyembunyikan informasi. Ketiga, manipulasi pixel pada gambar dapat
    dilakukan dengan berbagai teknik. Namun, tantangan utama dalam steganografi
    gambar adalah menemukan keseimbangan antara kapasitas penyembunyian data,
    kualitas gambar, dan ketahanan terhadap deteksi \cite{fikri2022optimasi}.

    Penelitian sebelumnya telah menunjukkan kemajuan signifikan dalam meningkatkan
    kapasitas penyembunyian data pada gambar grayscale. Kingsley et al.
    \cite{kingsley2020improving} berhasil meningkatkan kapasitas penyembunyian data
    hingga 450\% menggunakan metode code base. Meski begitu, penggunaan gambar
    grayscale membatasi aplikasi praktis karena hanya memiliki satu channel
    pencahayaan. Di sisi lain, Fikri et al. \cite{fikri2022optimasi}
    mendemonstrasikan bahwa penerapan steganografi pada gambar berwarna memiliki
    ketahanan yang baik, meskipun gambar dengan dominasi warna hitam menunjukkan
    kompatibilitas yang kurang optimal. Berdasarkan temuan-temuan tersebut,
    penelitian ini bertujuan untuk memperluas metode embedding ke gambar berwarna
    yang memiliki tiga channel warna (Merah, Hijau, dan Biru). Fokus utama adalah
    untuk meningkatkan kapasitas penyembunyian data sambil mempertahankan kualitas
    visual dan ketahanan terhadap deteksi. Penelitian ini juga akan mengevaluasi
    kompatibilitas metode yang diusulkan dengan berbagai jenis gambar berwarna,
    termasuk yang memiliki area gelap yang luas. Dengan mengoptimalkan penggunaan
    ketiga channel warna, diharapkan dapat dicapai peningkatan signifikan dalam
    kapasitas penyembunyian data dibandingkan dengan metode yang hanya menggunakan
    gambar grayscale, sambil tetap mempertahankan kealamiahan gambar yang
    dihasilkan.

    Aspek pengetahuan dalam penelitian ini berfokus pada teknik steganografi untuk
    gambar berwarna, metode steganografi berbasis kode, dan analisis kualitas
    gambar digital. Hal ini mencakup pemahaman mendalam tentang manipulasi channel
    warna RGB, teknik penyisipan data yang efektif, serta metode evaluasi kualitas
    gambar stego yang dihasilkan. Penekanan khusus diberikan pada optimasi
    penggunaan ketiga channel warna untuk memaksimalkan kapasitas penyimpanan data
    sambil mempertahankan kualitas visual.

    Dari sisi pengguna, penelitian ini ditujukan untuk beberapa kelompok utama.
    Pertama, profesional keamanan informasi yang membutuhkan solusi penyembunyian
    data yang andal dan aman. Kedua, peneliti di bidang steganografi dan keamanan
    data yang dapat memanfaatkan metode ini untuk pengembangan lebih lanjut.
    Ketiga, pengembang aplikasi keamanan data yang memerlukan komponen steganografi
    yang efisien dan mudah diintegrasikan.

    Aspek kegunaan penelitian mencakup berbagai aplikasi praktis dalam sistem data
    hiding, perlindungan hak cipta, dan watermarking digital. Metode yang
    dikembangkan dirancang dengan mempertimbangkan kemudahan implementasi,
    efisiensi komputasi, dan fleksibilitas penggunaan. Fokus utama adalah mencapai
    keseimbangan optimal antara kapasitas penyembunyian data, kualitas gambar
    hasil, dan ketahanan terhadap berbagai teknik deteksi steganografi.

    Infrastruktur yang digunakan dalam penelitian ini berbasis Mac mini M2 dengan
    spesifikasi teknis yang mendukung pemrosesan gambar yang efisien. Sistem
    dilengkapi dengan prosesor Apple M2 chip yang memiliki 8-core CPU dan 10-core
    GPU, didukung oleh 8GB memori dan penyimpanan SSD 256GB. Penggunaan sistem
    operasi macOS memberikan platform yang stabil untuk pengembangan dan pengujian
    metode steganografi yang diusulkan. }

\section{Literature Review}
\justify{
    % Penelitian sebelumnya telah mengeksplorasi berbagai metode steganografi. Kingsley et al. \cite{kingsley2020improving} menggunakan teknik multiple embedding berbasis kode pada gambar grayscale dan berhasil meningkatkan kapasitas hingga 450\% dengan PSNR 51 dB, namun terbatas pada grayscale. Penelitian lain seperti modifikasi GifShuffle oleh Andika et al. \cite{andika2020modifikasi} meningkatkan kapasitas pada GIF tetapi kompleks. Kombinasi LSB dengan RSA oleh Susanto et al. \cite{susanto2020kombinasi} atau dengan Caesar Cipher dan RC4 oleh Wiranata et al. \cite{wiranata2021aplikasi} meningkatkan keamanan namun menambah kompleksitas atau ukuran file. Metode LSB standar, seperti yang diteliti Basri et al. \cite{basri2021penerapan} dan Abdillah et al. \cite{abdillah2023implementasi}, efektif menjaga kualitas visual tetapi terbatas kapasitasnya dan rentan manipulasi. Secara umum, metode yang ada menunjukkan trade-off antara kapasitas, kualitas visual (PSNR), dan ketahanan, serta seringkali terbatas pada format gambar tertentu (misalnya grayscale atau GIF). Penelitian ini mencoba mengatasi keterbatasan kapasitas pada metode berbasis kode sebelumnya dengan menerapkan multiple embedding dan Polar Codes pada gambar RGB.

    Kingsley et al (2020) pada papernya membahas tentang rendahnya kapasitas
    penyisipan dan nilai PSNR (Peak Signal-to-Noise Ratio) pada skema steganografi
    berbasis kode yang ada, yang hanya mencapai 150\% kapasitas dan 48 dB kualitas
    visual gambar stegano. Untuk mengatasi masalah ini, metode yang diusulkan
    adalah teknik penyisipan ganda (multiple embedding), yang bertujuan untuk
    menyisipkan bit rahasia lebih dari sekali pada LSB (Least Significant Bit) dari
    piksel yang dipilih berdasarkan kunci rahasia. Parameter yang digunakan untuk
    mengukur keberhasilan metode ini meliputi kapasitas penyisipan dan nilai PSNR
    dari gambar stego yang dihasilkan. Kelebihan dari metode ini adalah kemampuan
    untuk mencapai kapasitas penyisipan yang lebih tinggi hingga 450\% dan nilai
    PSNR yang meningkat menjadi 51 dB, serta meningkatkan ketahanan terhadap
    serangan seperti scratching dan kompresi JPEG. Namun, penelitian ini masih
    diterapkan pada gambar hitam putih \cite{kingsley2020improving}.

    Pada penelitian yang dilakukan oleh Dwi Andika et al (2020), mereka membahas
    keterbatasan kapasitas penyimpanan dalam teknik steganografi menggunakan
    algoritma GifShuffle pada citra GIF. Tantangan utama yang dihadapi adalah
    kesulitan dalam menyisipkan pesan teks dengan ukuran besar. Untuk mengatasi
    masalah ini, mereka memodifikasi nilai bit dalam penyimpanan pesan, yang
    bertujuan meningkatkan kapasitas tanpa mengorbankan kualitas visual gambar. Uji
    coba menggunakan parameter seperti ukuran data yang dapat disisipkan serta
    kualitas gambar menunjukkan bahwa metode ini berhasil meningkatkan kapasitas
    penyimpanan hingga lebih dari 256 KB tanpa mengurangi kualitas gambar secara
    signifikan. Namun, metode ini memerlukan modifikasi tambahan pada algoritma
    GifShuffle yang dapat meningkatkan kompleksitas implementasi dan pemrosesan
    \cite{andika2020modifikasi}.

    Susanto et al (2020) dalam penelitiannya menggabungkan metode steganografi
    Least Significant Bit (LSB) dengan algoritma enkripsi RSA untuk meningkatkan
    keamanan dalam penyisipan pesan terenkripsi pada gambar. Parameter yang
    digunakan untuk mengukur kualitas gambar setelah penyisipan adalah Peak
    Signal-to-Noise Ratio (PSNR), dengan hasil yang menunjukkan PSNR tertinggi
    sebesar 78 dB untuk pesan berukuran 1024 bit. Kelebihan metode ini adalah
    perubahan kualitas gambar yang hampir tidak terlihat, namun kekurangannya
    adalah kompleksitas yang meningkat seiring dengan ukuran pesan dan besarnya
    kunci RSA yang digunakan, yang memengaruhi performa
    \cite{susanto2020kombinasi}.

    Basri et al (2021) meneliti penggunaan metode steganografi dengan teknik Least
    Significant Bit (LSB) untuk menyembunyikan gambar di dalam gambar lainnya,
    khususnya dalam konteks interaksi sosial melalui media digital. Penelitian ini
    mengukur rasio ukuran gambar tersembunyi terhadap gambar cover serta kemampuan
    gambar cover mempertahankan kualitas visualnya. Hasilnya menunjukkan bahwa
    metode ini efektif menyembunyikan informasi tanpa perubahan signifikan pada
    gambar cover. Namun, metode ini memiliki keterbatasan dalam menangani gambar
    dengan transparansi, karena hanya memperhitungkan komponen warna merah, hijau,
    dan biru (RGB), tanpa memperhitungkan komponen alpha \cite{basri2021penerapan}.

    Penelitian Wiranata et al (2021) berfokus pada penyisipan pesan rahasia dalam
    gambar dan audio menggunakan metode Least Significant Bit (LSB) yang
    dikombinasikan dengan enkripsi Caesar Chipper dan Rivest Code 4 untuk menjaga
    kerahasiaan data. Penelitian ini menguji kemampuan aplikasi untuk
    menyembunyikan dan mengambil pesan secara utuh, serta perubahan ukuran file
    gambar dan audio setelah proses penyisipan. Kelebihan metode ini adalah
    kemampuannya menjaga kerahasiaan tanpa perubahan signifikan pada kualitas
    gambar atau suara. Namun, ada perubahan ukuran file yang diakibatkan oleh
    proses enkripsi dan penyisipan pesan \cite{wiranata2021aplikasi}.

    Penelitian yang dilakukan oleh Abdillah et al (2023) membahas tentang
    perlindungan data dan akses dengan menggunakan steganografi melalui teknik
    Least Significant Bit (LSB), di mana teks disisipkan dalam gambar melalui
    perubahan nilai pixel terkecil. Parameter yang diukur mencakup akurasi
    penyisipan teks dan kualitas gambar yang tetap terjaga setelah proses encoding
    dan decoding. Kelebihan utama metode ini adalah kemampuannya menyembunyikan
    informasi tanpa mengubah kualitas visual gambar secara signifikan. Namun,
    metode ini memiliki keterbatasan dalam kapasitas penyisipan teks dan rentan
    terhadap serangan manipulasi gambar yang dapat merusak data tersembunyi
    \cite{abdillah2023implementasi}.

    Tabel \ref{tab:comparison} menunjukkan perbandingan hasil penelitian terdahulu.

    \begin{table}[h!]
        \caption{Comparison of Previous Research}
        \begin{tabularx}{\textwidth}{|p{2cm}|X|p{2.5cm}|p{2.5cm}|}
            \hline
            \textbf{Author}        & \textbf{Method}                                                    & \textbf{Advantages}                                      & \textbf{Disadvantages}                                  \\
            \hline
            Kingsley et al. (2020) & Multiple embedding technique with secret key-based pixel selection & Increased capacity up to 450\%, PSNR improved to 51dB    & Limited to grayscale images only                        \\
            \hline
            Andika et al. (2020)   & Modified GifShuffle algorithm for GIF images                       & Increased storage capacity over 256KB                    & Complex implementation, limited to GIF format           \\
            \hline
            Susanto et al. (2020)  & Combined LSB with RSA encryption                                   & High PSNR (78dB) for 1024-bit messages                   & Increased complexity with larger messages and RSA keys  \\
            \hline
            Basri et al. (2021)    & LSB technique for image-in-image hiding                            & Effective information hiding with minimal visual changes & Limited handling of transparency (alpha channel)        \\
            \hline
            Wiranata et al. (2021) & LSB combined with Caesar Cipher and RC4                            & Good security with dual encryption                       & File size increases after embedding                     \\
            \hline
            Abdillah et al. (2023) & LSB technique for text embedding                                   & Maintains visual quality effectively                     & Limited text capacity, vulnerable to image manipulation \\
            \hline
        \end{tabularx}
        \label{tab:comparison}
    \end{table}

    Berdasarkan tabel \ref{tab:comparison}, dapat disimpulkan bahwa berbagai metode
    steganografi telah menunjukkan kelebihan dan kekurangan masing-masing. Metode
    multiple embedding oleh Kingsley et al. (2020) berhasil meningkatkan kapasitas
    penyembunyian data secara signifikan hingga 450\% dengan kualitas visual yang
    baik (PSNR 51 dB), namun terbatas pada gambar grayscale. Metode GifShuffle yang
    dimodifikasi oleh Andika et al. (2020) meningkatkan kapasitas penyimpanan pada
    gambar GIF, tetapi kompleksitas implementasinya tinggi. Kombinasi LSB dan
    enkripsi RSA oleh Susanto et al. (2020) menghasilkan PSNR tinggi (78 dB) namun
    dengan peningkatan kompleksitas. Metode LSB oleh Basri et al. (2021) efektif
    untuk menyembunyikan gambar dalam gambar lain dengan perubahan visual minimal,
    tetapi kurang optimal untuk gambar dengan transparansi. Kombinasi LSB dengan
    Caesar Cipher dan RC4 oleh Wiranata et al. (2021) memberikan keamanan yang baik
    namun meningkatkan ukuran file. Terakhir, metode LSB oleh Abdillah et al.
    (2023) efektif dalam menjaga kualitas visual namun memiliki keterbatasan
    kapasitas teks dan rentan terhadap manipulasi gambar.

    Artikel ini memperkenalkan skema steganografi kuantum yang menyembunyikan citra
    abu-abu di dalam citra berwarna. Keamanan ditingkatkan dengan menggunakan
    pengacakan Arnold dan menyematkan citra rahasia dalam tiga cara berbeda: DWT di
    saluran R, DCT di saluran G, dan LSB di saluran B. Pendekatan ini meningkatkan
    ketidaktampakan, keamanan, dan ketahanan steganografi. PSNR rata-rata adalah
    54,33 dB, menunjukkan kualitas visual yang tinggi, dan skema ini dapat
    menangani citra rahasia hingga setengah ukuran citra sampul. Namun, artikel
    tersebut mengakui kurangnya dukungan perangkat keras untuk komputer kuantum dan
    bergantung pada simulasi komputer klasik untuk pengujian.\cite{dong2024new}

}

\section{Problem Statement}
\justify{
    % Metode steganografi berbasis kode sebelumnya, seperti yang dikembangkan oleh Kingsley et al. \cite{kingsley2020improving}, telah berhasil meningkatkan kapasitas penyembunyian data secara signifikan hingga 450\%. Namun, implementasinya pada gambar grayscale membatasi aplikasinya secara luas \cite{soetarmono2012studi} karena sebagian besar gambar digital yang digunakan saat ini adalah gambar berwarna (RGB) yang memiliki tiga channel warna. Terdapat kebutuhan untuk mengembangkan metode steganografi berbasis kode yang dapat memanfaatkan potensi penuh gambar RGB untuk mencapai kapasitas penyembunyian yang lebih tinggi sambil tetap menjaga kualitas visual yang baik dan ketahanan terhadap deteksi.

    Penelitian sebelumnya dalam steganografi berbasis kode telah menunjukkan
    kemajuan signifikan. Kingsley et al. \cite{kingsley2020improving} berhasil
    meningkatkan kapasitas penyembunyian data dari sekitar 150\% menjadi 450\%,
    yang merupakan peningkatan substansial untuk case yang membutuhkan kapasitas
    tinggi. Meskipun demikian, penelitian pada gambar grayscale
    \cite{soetarmono2012studi} masih membatasi pengaplikasiannya karena kurangnya
    fleksibilitas dalam penggunaan channel warna. }

\section{Objective and Hypotheses}
\justify{
    Tujuan utama penelitian ini adalah:
    \begin{enumerate}
        \item Mengembangkan teknik steganografi berbasis kode dengan multiple embedding yang
              dioptimalkan untuk gambar RGB, dengan target peningkatan kapasitas
              penyembunyian data minimal 50\% dibandingkan metode sebelumnya (Kingsley et al.
              \cite{kingsley2020improving}). Hipotesisnya adalah penggunaan Polar Codes pada
              gambar RGB akan memungkinkan peningkatan kapasitas yang lebih besar daripada
              aplikasi pada gambar grayscale.
        \item Mempertahankan kualitas visual gambar stego dengan nilai Peak Signal-to-Noise
              Ratio (PSNR) di atas 40 dB. Hipotesisnya adalah peningkatan kapasitas melalui
              multiple embedding dan Polar Codes tidak akan secara signifikan menurunkan
              kualitas visual gambar \cite{nasution2018image}.
    \end{enumerate}
    Metode yang diusulkan, yaitu multiple embedding dengan Polar Codes, dipilih karena potensi Polar Codes dalam mencapai kapasitas Shannon dengan kompleksitas yang relatif rendah dan kemampuannya dalam koreksi kesalahan, yang diharapkan dapat menyeimbangkan antara kapasitas dan kualitas visual pada gambar RGB.
}

\section{Research Method}
\subsection{Method requirement specification }
\justify{
    Kebutuhan utama metode yang dikembangkan adalah:
    \begin{enumerate}
        \item Mampu meningkatkan kapasitas penyembunyian data minimal 50\% dibandingkan
              metode Kingsley et al. \cite{kingsley2020improving}.
        \item Menjaga kualitas visual gambar stego dengan PSNR di atas 40 dB.
        \item Mengimplementasikan algoritma penyisipan multiple embedding yang efektif untuk
              gambar berwarna RGB.
    \end{enumerate}
    Sistem harus dapat memproses gambar input (secret image) dan gambar penampung (cover image) dalam format RGB, melakukan proses encoding/embedding dan decoding/extraction, serta tahan terhadap serangan umum.
}

\subsection{Design and Implementation of the proposed method }
\justify{

    \begin{figure}[h]
        \centering
        \includegraphics[width=0.8\textwidth]{gambar/proposed.png}
        \caption{Design method}
        \label{fig:flow}
    \end{figure}

    Desain metode yang diusulkan melibatkan beberapa tahapan kunci seperti yang
    diilustrasikan pada Gambar 1. Proses utamanya adalah menyisipkan bit-bit
    rahasia dari 'Secret Image' ke dalam 'Cover Color Image' menggunakan Polar
    Codes. Polar Codes digunakan untuk mengkodekan data rahasia sebelum penyisipan
    untuk meningkatkan reliabilitas dan mencapai kapasitas yang tinggi dengan
    kompleksitas $O(N \log N)$. Posisi piksel untuk penyisipan ditentukan secara
    acak namun deterministik berdasarkan seed. Selain itu, digunakan mekanisme Key
    Trace yang disisipkan ke dalam 'Agreed Image' untuk menghasilkan 'Agreed
    Stego', serta skema pembagian kunci (Shamir's Secret Sharing) di mana 'Share 2'
    disisipkan ke 'Stego Image' dan 'Share 1' disimpan terpisah. Implementasi
    dilakukan menggunakan Python dengan library OpenCV dan NumPy, mendukung gambar
    RGB dan grayscale untuk perbandingan. Fokus implementasi adalah pada mekanisme
    multiple embedding, adaptasi Polar Codes untuk RGB, optimasi kapasitas
    ($>$50\%), dan preservasi kualitas (PSNR $>$ 40 dB). }

\subsection{Experimental results }
\justify{
    Tahap eksperimen dirancang untuk mengevaluasi tiga aspek utama: Imperceptibility, Robustness, dan Capacity, seperti pada Gambar 2 proposal. Imperceptibility diukur menggunakan Mean Square Error (MSE) dan Peak Signal-to-Noise Ratio (PSNR) untuk membandingkan kualitas visual gambar stego dengan gambar asli. Robustness diuji terhadap berbagai serangan seperti Noise (Gaussian \cite{selamiEffectGaussian}, Salt \& Pepper \cite{kaisar2008salt}, Speckle \cite{mansourpour2006effects}), Cropping, Scratching (Single \& Multiple), dan Kompresi JPEG \cite{reddy2015medical}, dengan mengukur PSNR gambar rahasia yang berhasil dipulihkan. Capacity dievaluasi dengan mengukur rasio bit rahasia yang dapat disematkan terhadap ukuran gambar cover (bpp), menggunakan variasi ukuran secret image dan karakteristik gambar cover. Hasil eksperimen metode yang diusulkan (Polar Codes dengan multiple embedding) akan dibandingkan dengan metode sebelumnya (Reed-Muller code dengan multiple embedding). Saat ini, pengujian awal sedang dipersiapkan seiring dengan penyelesaian tahap implementasi.
}

\subsection{Analysis of the experimental results}
\justify{
    Analisis hasil eksperimen akan difokuskan pada perbandingan kuantitatif antara metode yang diusulkan (Polar Codes) dan metode acuan (Reed-Muller). Metrik utama yang akan dianalisis adalah:
    \begin{itemize}
        \item Imperceptibility: Nilai MSE dan PSNR akan dihitung dan dibandingkan. PSNR yang
              lebih tinggi (di atas 40 dB) dan MSE yang lebih rendah menunjukkan kualitas
              visual yang lebih baik.
        \item Robustness: Nilai PSNR dari gambar rahasia yang dipulihkan setelah berbagai
              serangan akan dianalisis untuk mengukur ketahanan metode. Kemampuan koreksi
              kesalahan (ECC) dari kode yang digunakan (Polar Codes vs Reed-Muller) akan
              dievaluasi berdasarkan tingkat keberhasilan pemulihan data.
        \item Capacity: Kapasitas penyisipan (EC) dalam bit per pixel (bpp) akan dihitung
              menggunakan persamaan (6) dan dibandingkan antar metode. Analisis akan mencakup
              pengaruh variasi histogram gambar dan ukuran secret image terhadap kapasitas.
    \end{itemize}
    Analisis statistik seperti uji-t atau ANOVA mungkin digunakan untuk mengevaluasi signifikansi perbedaan kinerja antar metode.
}

\section{Schedule Realization}

\begin{table}[h]
    \centering
    \begin{tabular}{|l|c|c|c|c|l|l|}
        \hline
        \rowcolor[HTML]{FFFFFF}
        \multicolumn{1}{|c|}{\cellcolor[HTML]{FFFFFF}\multirow{2}{*}{Activity}} & \multicolumn{2}{c|}{\cellcolor[HTML]{FFFFFF}Schedule} & \multicolumn{2}{c|}{\cellcolor[HTML]{FFFFFF}Realization} & \multicolumn{1}{c|}{\cellcolor[HTML]{FFFFFF}\multirow{2}{*}{Output Target}} & \multicolumn{1}{c|}{\cellcolor[HTML]{FFFFFF}\multirow{2}{*}{Realization}}                                                                                                 \\ \cline{2-5}
        \rowcolor[HTML]{FFFFFF}
        \multicolumn{1}{|c|}{\cellcolor[HTML]{FFFFFF}}                          & \multicolumn{1}{c|}{\cellcolor[HTML]{FFFFFF}Start}    & \multicolumn{1}{c|}{\cellcolor[HTML]{FFFFFF}End}         & \multicolumn{1}{c|}{\cellcolor[HTML]{FFFFFF}Start}                          & \multicolumn{1}{c|}{\cellcolor[HTML]{FFFFFF}End}                          & \multicolumn{1}{c|}{\cellcolor[HTML]{FFFFFF}} & \multicolumn{1}{c|}{\cellcolor[HTML]{FFFFFF}} \\ \hline
        \rowcolor[HTML]{EFEFEF}
        Literature study                                                        & Sem 1                                                 & Sem 1                                                    & Sem 1                                                                       & Sem 1                                                                     & Proposal Bab 1, 2                             & Selesai                                       \\ \hline
        Problem identification                                                  & Sem 1                                                 & Sem 1                                                    & Sem 1                                                                       & Sem 1                                                                     & Proposal Bab 3                                & Selesai                                       \\ \hline
        \rowcolor[HTML]{EFEFEF}
        Contribution formulation                                                & Sem 1                                                 & Sem 1                                                    & Sem 1                                                                       & Sem 1                                                                     & Proposal Bab 4                                & Selesai                                       \\ \hline
        Hypothesis formulation                                                  & Sem 1                                                 & Sem 1                                                    & Sem 1                                                                       & Sem 1                                                                     & Proposal Bab 4                                & Selesai                                       \\ \hline
        \rowcolor[HTML]{EFEFEF}
        Proposal                                                                & Sem 1                                                 & Sem 1                                                    & Sem 1                                                                       & Sem 1                                                                     & Dokumen Proposal                              & Selesai                                       \\ \hline
        Design Encoding Process                                                 & Sem 2                                                 & Sem 2                                                    & Sem 2                                                                       & Sem 2                                                                     & Desain Algoritma                              & Selesai                                       \\ \hline
        \rowcolor[HTML]{EFEFEF}
        Implement embedding schema                                              & Sem 2                                                 & Sem 3                                                    & Sem 2                                                                       & \blue Sem 3                                                               & Kode Program (Embedding)                      & \blue On Progress                             \\ \hline
        Design decoding process                                                 & Sem 2                                                 & Sem 2                                                    & Sem 2                                                                       & \blue Sem 2                                                               & Desain Algoritma                              & \blue On Progress                             \\ \hline
        \rowcolor[HTML]{EFEFEF}
        Design sharing schema                                                   & Sem 2                                                 & Sem 3                                                    & Sem 2                                                                       & \blue Sem 3                                                               & Kode Program (Sharing)                        & \blue On Progress                             \\ \hline
        Imperceptibility testing                                                & Sem 3                                                 & Sem 3                                                    & Sem 3                                                                       & -                                                                         & Hasil Uji PSNR/MSE                            & Belum Mulai                                   \\ \hline
        \rowcolor[HTML]{EFEFEF}
        Robustness testing                                                      & Sem 3                                                 & Sem 4                                                    & Sem 3                                                                       & -                                                                         & Hasil Uji Serangan                            & Belum Mulai                                   \\ \hline
        Capacity testing                                                        & Sem 3                                                 & Sem 4                                                    & Sem 3                                                                       & -                                                                         & Hasil Uji Kapasitas                           & Belum Mulai                                   \\ \hline
        \rowcolor[HTML]{EFEFEF}
        Thesis draft                                                            & Sem 4                                                 & Sem 4                                                    & Sem 4                                                                       & -                                                                         & Draft Tesis Lengkap                           & Belum Mulai                                   \\ \hline
    \end{tabular}
    \caption{Realisasi Jadwal per 18 April 2025}
\end{table}

% Tambahkan bagian Daftar Pustaka
\newpage % Mulai halaman baru untuk daftar pustaka jika diinginkan
\begin{thebibliography}{99} % Angka 99 adalah placeholder untuk label terpanjang

    \bibitem{pujianto2021uji}
    Ikwan Pujianto. "Uji Ketahanan Citra Digital Terhadap Manipulasi Robustness Pada Steganography". \textit{Jurnal Informatika dan Rekayasa Perangkat Lunak} 2.1 (2021), pp. 16-27.

    \bibitem{siaulhak2023sistem}
    Siaulhak Siaulhak, Safwan Kasma et al. "Sistem Pengiriman File Menggunakan Steganografi Pengolahan Citra Digital Berbasis Matriks Laboratory". \textit{BANDWIDTH: Journal of Informatics and Computer Engineering} 1.2 (2023), pp. 75-81.

    \bibitem{soetarmono2012studi}
    Anggya ND Soetarmono. "Studi Mengenai Aplikasi Steganografi Camouflage". \textit{Teknika} 1.1 (2012), pp. 55-65.

    \bibitem{fikri2022optimasi}
    Muhammad Alfin Fikri and FX Ferdinandus. "Optimasi Teknik Steganografi Amelsbr Pada Empat Bit Terakhir Dengan Cover Image Berwarna". \textit{Antivirus: Jurnal Ilmiah Teknik Informatika} 16.1 (2022), pp. 25-38.

    \bibitem{kingsley2020improving}
    Katandawa Alex Kingsley and Ari Moesriami Barmawi. "Improving Data Hiding Capacity in Code Based Steganography using Multiple Embedding." In: (). % Mungkin perlu dilengkapi info publikasi jika ada

    \bibitem{andika2020modifikasi}
    Dwi Andika and Dedi Darwis. "Modifikasi Algoritma Gifshuffle Untuk Peningkatan Kualitas Citra Pada Steganografi". \textit{Jurnal Ilmiah Infrastruktur Teknologi Informasi} 1.2 (2020), pp. 19-23.

    \bibitem{susanto2020kombinasi}
    Ajib Susanto and Ibnu Utomo Wahyu Mulyono. "Kombinasi LSB-RSA untuk Peningkatan Imperceptibility pada Kripto-Stegano Gambar RGB". In: (2020). % Mungkin perlu dilengkapi info publikasi jika ada

    \bibitem{basri2021penerapan}
    Muh Basri and Muhammad Fadhlil Gushari. "Penerapan Steganografi Gambar Berwarna pada Delapan Image Cover Menggunakan Metode LSB". \textit{Jurnal Sintaks Logika} 1.3 (2021), pp. 153-158.

    \bibitem{wiranata2021aplikasi}
    Ade Davy Wiranata and Rima Tamara Aldisa. "Aplikasi Steganografi Menggunakan Least Significant Bit (LSB) dengan Enkripsi Caesar Chipper dan Rivest Code 4 (RC4) Menggunakan Bahasa Pemrograman JAVA". \textit{Jurnal JTIK (Jurnal Teknologi Informasi dan Komunikasi)} 5.3 (2021), pp. 277-281.

    \bibitem{abdillah2023implementasi}
    Muhammad Oemar Abdillah, Ogie Ariansah Pane and Farhan Rusdy Asyhary Lubis. "Implementasi Keamanan Aset Informasi Steganografi Menggunakan Metode Least Significant Bit (LSB)". \textit{Jurnal Sains dan Teknologi (JSIT)} 3.1 (2023), pp. 40-46.

    \bibitem{juneja2013improved}
    Mamta Juneja and Parvinder S Sandhu. "An improved LSB based steganography technique for RGB color images". \textit{International journal of computer and communication engineering} 2.4 (2013), p. 513.

    \bibitem{nasution2018image}
    AB Nasution, S Efendi and S Suwilo. "Image steganography in securing sound file using arithmetic coding algorithm, triple data encryption standard (3DES) and modified least significant bit (MLSB)". \textit{Journal of Physics: Conference Series}. Vol. 1007. 1. IOP Publishing. 2018, p. 012010.

    \bibitem{selamiEffectGaussian}
    Ameen Mohammed Abd-Alsalam Selami and Ahmed Freidoon Fadhil. "A Study of the Effects of Gaussian Noise on Image". In: (). % Mungkin perlu dilengkapi info publikasi jika ada

    \bibitem{kaisar2008salt}
    Shakair Kaisar and Jubayer AI Mahmud. "Salt and Pepper Noise Detection and removal by Tolerance based selective Arithmetic Mean Filtering Technique for image restoration". \textit{International Journal of computer science and network security} 8.6 (2008), pp. 271-278.

    \bibitem{mansourpour2006effects}
    Mostafa Mansourpour, MA Rajabi and JAR Blais. "Effects and performance of speckle noise reduction filters on active radar and SAR images". In: Proc. Isprs. Vol. 36. 1. 2006, W41. % Mungkin perlu dilengkapi info publikasi jika ada

    \bibitem{reddy2015medical}
    Mr Venugopal Reddy CH et al. "Medical image watermarking schemes against salt and pepper noise attack". \textit{International Journal of Bio-Science and Bio-Technology} 7.6 (2015), pp. 55-64.

\end{thebibliography}

%\section{Summary}
%\supervisorcomments
\end{document}