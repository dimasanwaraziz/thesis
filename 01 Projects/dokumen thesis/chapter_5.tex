\chapter{CONCLUSION AND RECOMMENDATIONS}

%========================================================================
\section{Conclusions}
%========================================================================
These are brief, generalized statements in answer to the general and each of the specific sub-problems. These contain generalized in relation to the population. These are general inferences applicable to a wider and similar population. Flexibility is considered in making of conclusions. It is not a must to state conclusions on a one-to-one correspondence with the problems and the findings as all variables can be subsume in one paragraph. Conclusions may be used as generalizations from a micro to a macro-level or vice versa (ZOOM LENS approach). \textbf{Conclusions should be written on paragraph}.

%========================================================================
\section{Recommendations}
%========================================================================
They should be based on the findings and conclusion of the study. Recommendations may be specific or general or both. They may include suggestions for further studies. They should be in non-technical language, feasible, workable, flexible, doable, and adaptable. An action plan is optional.
