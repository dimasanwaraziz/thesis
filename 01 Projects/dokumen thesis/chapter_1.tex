\chapter{INTRODUCTION}
This chapter includes the following subtopics, namely: (1) Rationale; (2) Theoretical Framework; (3) Conceptual Framework/Paradigm; (4) Statement of the problem; (5) Hypothesis (Optional); (6) Assumption (Optional); (7) Scope and Delimitation;  and (8) Importance of the study.

% =========================================================
\section{Rationale}
% =========================================================
This section have to explain: (a) the background of the study; (b) describe the problem situation considering global, national and local forces; (c) Justify the existence of the problem situation by citing statistical data and authoritative sources; and (d) Make a clinching statement that will relate the background to the proposed research problem.

% =========================================================
\section{Theoretical Framework}
% =========================================================
Discuss the theories and/or concepts, which are useful in conceptualizing the research.

% =========================================================
\section{Conceptual Framework/Paradigm}
% =========================================================
Identify and discuss the variables related to the problem, and present a schematic diagram of the paradigm of the research and discuss the relationship of the elements/variables therein.

% =========================================================
\section{Statement of the Problem}
% =========================================================
The general problem must be reflective of the title. It should be stated in such a way that it is not answerable by yes or no, not indicative of when and where. Rather, it should reflect between and among variables. Each sub-problem should cover mutually exclusive dimensions (no overlapping). The sub-problem should be arranged in logical order from actual to analytical following the flow in the research paradigm.

% =========================================================
\section{Objective and Hypotheses}
% =========================================================
First, explain the objective according the problem, then hypothesis. A hypothesis should be measurable/ desirable. It expresses expected relationship between two or more variables. It is based on the theory and/or empirical evidence. There are techniques available to measure or describe the variables. It is on a one to one correspondence with the specific problems of the study. A hypothesis in statistical form has the following characteristics: (a) it is used when the test of significance of relationships and difference of measures are involved; and (b) the level of significance if stated.

% =========================================================
\section{Assumption}
% =========================================================
An assumption should be based on the general and specific problems. It is stated in simple, brief, generally accepted statement.

% =========================================================
\section{Scope and Delimitation}
% =========================================================
Indicate the principal variables, locale, timeframe, and justification.

% =========================================================
\section{Significance of the Study}
% =========================================================
This section describes the contributions of the study as new knowledge, make findings more conclusive. It cites the usefulness of the study to the specific groups.
