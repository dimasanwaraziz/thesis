\documentclass{ittelkom}
\usepackage{graphicx}
\usepackage{multirow}
\usepackage{tabularx, ragged2e, lmodern, bigstrut}
\usepackage[sgvnames, table]{xcolor}
\newcommand{\blue}{\cellcolor{blue!75}}


\begin{document}

\project{Dynamic Path Planning}
\submit{Judhi Santoso}
\nim{898898989}
\concentration{Algorithm Performance}
\email{judhi@yahoo.com}
\supervisor{Pembimbing 1}{Pembimbing 2}
\submitdate{14}{11}{2021}

\summary{Topic Summary}
{This is a summary and a word length of 300 words. Summary is written briefly from the entire contents of the thesis/proposal and so on until it is finished.}

\newpage
\section{INTRODUCTION \color{red}(assessment 1)}
Introduction section provides a description of the basic facts and importance of the research area - What is your research area, the motivation of research, and how important is it for the industry practice/knowledge advancement?


\justify{Starting with providing reasons of why phenomenon is chosen as the research topic; possibility that the phenomenon will give new concepts as a result;  }

Afterward, provide a brief outline of the most important studies that have been conducted so far related to above mentioned phenomenon.

At the end of the section, provide brief statement stating {\bf the importance of this phenomenon to be discussed} by showing the gap of existing condition and the future condition.

To complete this section, see the detail in the assignment form of assessment-1.

\section{Preliminary Literature Review \color{red}(assessment 2)}
Preliminary literature review: provide a summary of previous related research on the research problem and their strength and weakness and a justification of your research - What is known/what have been done by others? And, why your research is still necessary?

To complete this section, see the detail in the assignment form of assessment-2.

\section{Problem Statement \color{red} (assessment 2)} \label{resquest}
Problem statement concludes the challenge issues/open problems based on {\bf the strength} and limitation emerged on the previous section. It provides a clear and concise description of the issues that need to be addressed - What is the specific problem in that research area that you will address (e.g. lack of understanding of a subject, low performance ...)?

\section{Objective and Hypothesis \color{red}(assessment 3)} \label{hyp}
This section should contain objective research direction (sharp and measurable) and the hypotheses (The explanation of method, concept which will be used to solve the problem as well as the reason why they will be used in the research; the explanation about the difference between method, concept, and theorem which will be used and the method, concept method, concept which were used previously.

To complete this section, see the detail in the assignment form of assessment-3. 

%\begin{figure}
%\centering
%\includegraphics[width=0.4\textwidth]{TelU_1.jpg}
%%\caption{Logo University}
%\end{figure}


\section{Research Method \color{red}(assessment 4,5)}
Research method should contain brief description on fundamental knowledge related to the addressed problem. You should highlight that the concept is appropriate for your problem. Write down the mapping between the concept used and the problem solved.

Also, it defines the logic steps - What to do and how to solve the problem and achieve proposed objectives? Which research methods (e.g. survey, modeling, case study ...) will be used? 

More specifically, the section comprises the following components:
\begin{itemize}
\item[1.] Requirement identification
\item[2.] Design process (including general system architecture)
\item[3.] Implementation process
\item[4.] Experiment design and plan (including data collection process)
\item[5.] Analysis/Evaluation method which will be used for analyzing the experiment result
\end{itemize}

To complete this section, see the detail in the assignment form of assessment-4 and assessment-5.

\section{Work Plan and Time Schedule}
Write a work plan along with the schedule for completion. The following is the example. You may adjust the activities  and time schedule according to the problem.
\newline

\begin{table}[h!]
\caption{Activity Schedule (\color{red}example) \label{tab:schedule}}
\noindent\begin{tabularx}{\linewidth}{|>{\bfseries}l|l|*{11}{>{\centering\arraybackslash}X|}>{\centering\arraybackslash}X<{\bigstrut}|}
\hline
\multicolumn{2}{|l|}{}&\multicolumn{12}{c|}{\bfseries SEMESTER\bigstrut}\\
\cline{3-14}
\multicolumn{2}{|c|}{\bfseries Activity}&\multicolumn{3}{c|}{\bfseries 1}&\multicolumn{3}{c|}{\bfseries 2}&\multicolumn{3}{c|}{\bfseries 3}&\multicolumn{3}{c|}{\bfseries 4\bigstrut}\\
\hline

1&Literature study&\blue&&&&&&&&&&&\\
\hline
2&Problem identification&\blue&\blue&&&&&&&&&&\\
\hline
3&Contribution formulation&\blue&\blue&&&&&&&&&&\\
\hline
4&Hypothesis formulation&\blue&\blue&&&&&&&&&&\\
\hline
5&Proposal&&&\blue&&&&&&&&&\\
\hline
6&Data collection&&&&\blue&&&&&&&&\\
\hline
7&Requirement identification&&&&&&\blue&\blue&&&&&\\
\hline
8&Design process&&&&&&&&\blue&\blue&&&\\
\hline
9&Implementation process&&&&&&&&&\blue&&&\\
\hline
10&Experiment design&&&&&&&&&\blue&&&\\
\hline
11&Evaluation and analysis&&&&&&&&&\blue&\blue&\blue&\\
\hline
12&Thesis draft&&&&&&&&&&&&\blue\\
\hline

\end{tabularx}
\end{table}

%\section{Summary}
\supervisorcomments


\end{document} 
