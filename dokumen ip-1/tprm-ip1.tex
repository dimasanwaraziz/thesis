\documentclass{ittelkom}
\usepackage{graphicx}
\usepackage{multirow}
\usepackage{tabularx, ragged2e, lmodern, bigstrut}
\usepackage[sgvnames, table]{xcolor}
\usepackage{cite} % Tambahkan paket cite untuk penanganan sitasi yang lebih baik

\newcommand{\blue}{\cellcolor{blue!75}}

\begin{document}

\laporancapstonecover{April, 18}{2025}{Multiple Embedding steganography for RGB Image}{Dimas Anwar Aziz}{203022410012}{Prof. Ari Moesriami Barmawi, Ph.D.}{-}{dimasanwaraziz@student.telkomuniversity.ac.id}

\newpage
\normalsize
\tableofcontents

\newpage
\section*{Progress Summary: (max 500 words)}
\label{summary}
\justify{
    Penelitian mengenai steganografi Multiple Embedding untuk gambar RGB telah mencapai kemajuan. Tahapan studi literatur, identifikasi masalah, formulasi kontribusi, dan penyusunan proposal telah diselesaikan. Saat ini, fokus penelitian berada pada tahap desain dan implementasi metode yang diusulkan. Berdasarkan studi awal dan tinjauan literatur, teridentifikasi bahwa metode sebelumnya yang menggunakan pendekatan berbasis kode seperti Reed-Muller memiliki keterbatasan, terutama dalam hal kapasitas pada gambar grayscale \cite{kingsley2020improving}. Oleh karena itu, penelitian ini mengusulkan penggunaan Polar Codes yang dikombinasikan dengan teknik multiple embedding untuk gambar RGB. Tujuannya adalah meningkatkan kapasitas penyembunyian data secara signifikan (target minimal 50\% lebih tinggi dari metode Kingsley et al. \cite{kingsley2020improving}) sambil mempertahankan kualitas visual gambar stego (PSNR $>$ 40 dB).

    Proses desain metode, termasuk alur kerja encoding, pembangkitan jejak kunci,
    penyisipan kunci dan data rahasia menggunakan Polar Codes, serta skema
    pembagian kunci (secret sharing) telah dirancang. Tahap implementasi awal
    sedang berjalan, mencakup pengembangan fungsi-fungsi inti dalam Python
    menggunakan library seperti OpenCV dan NumPy. Tantangan utama saat ini adalah
    mengoptimalkan algoritma Polar Codes untuk penyisipan pada ketiga channel warna
    RGB secara efisien dan memastikan integrasi yang mulus dengan mekanisme
    multiple embedding. Tahap selanjutnya adalah pengujian awal imperceptibility
    dan kapasitas, diikuti dengan pengujian robustness terhadap berbagai serangan.
    Kesulitan yang dihadapi terutama berkaitan dengan kompleksitas matematis Polar
    Codes dan penyesuaiannya untuk domain steganografi gambar berwarna. Komentar
    dari pembimbing terkait metode penelitian akan terus diintegrasikan dalam
    pengembangan. Diharapkan tahap implementasi dan pengujian awal dapat
    diselesaikan sesuai jadwal. }

\section{Background} \label{background}
\justify{
    Keamanan data digital menjadi sangat penting di era digital saat ini untuk melindungi informasi sensitif \cite{pujianto2021uji, siaulhak2023sistem}. Kriptografi dan steganografi adalah dua solusi utama untuk tujuan ini. Steganografi menyembunyikan keberadaan pesan rahasia di dalam media lain, seperti gambar, tanpa terdeteksi \cite{pujianto2021uji}. Umumnya, pesan dienkripsi terlebih dahulu sebelum disembunyikan \cite{soetarmono2012studi}. Gambar digital populer digunakan sebagai media penyembunyian karena ketersediaannya yang luas, kapasitas penyembunyian yang inheren, dan kemudahan manipulasi piksel. Namun, tantangan utamanya adalah menyeimbangkan kapasitas, kualitas gambar, dan ketahanan terhadap deteksi \cite{fikri2022optimasi}. Penelitian ini berfokus pada peningkatan kapasitas steganografi pada gambar berwarna (RGB) yang memiliki tiga channel, melampaui keterbatasan metode sebelumnya yang seringkali fokus pada gambar grayscale. Dengan mengoptimalkan ketiga channel warna, diharapkan kapasitas dapat ditingkatkan signifikan sambil menjaga kualitas visual dan ketahanan.
}

\section{Literature Review}
\justify{
    Penelitian sebelumnya telah mengeksplorasi berbagai metode steganografi. Kingsley et al. \cite{kingsley2020improving} menggunakan teknik multiple embedding berbasis kode pada gambar grayscale dan berhasil meningkatkan kapasitas hingga 450\% dengan PSNR 51 dB, namun terbatas pada grayscale. Penelitian lain seperti modifikasi GifShuffle oleh Andika et al. \cite{andika2020modifikasi} meningkatkan kapasitas pada GIF tetapi kompleks. Kombinasi LSB dengan RSA oleh Susanto et al. \cite{susanto2020kombinasi} atau dengan Caesar Cipher dan RC4 oleh Wiranata et al. \cite{wiranata2021aplikasi} meningkatkan keamanan namun menambah kompleksitas atau ukuran file. Metode LSB standar, seperti yang diteliti Basri et al. \cite{basri2021penerapan} dan Abdillah et al. \cite{abdillah2023implementasi}, efektif menjaga kualitas visual tetapi terbatas kapasitasnya dan rentan manipulasi. Secara umum, metode yang ada menunjukkan trade-off antara kapasitas, kualitas visual (PSNR), dan ketahanan, serta seringkali terbatas pada format gambar tertentu (misalnya grayscale atau GIF). Penelitian ini mencoba mengatasi keterbatasan kapasitas pada metode berbasis kode sebelumnya dengan menerapkan multiple embedding dan Polar Codes pada gambar RGB.
}

\section{Problem Statement}
\justify{
    Metode steganografi berbasis kode sebelumnya, seperti yang dikembangkan oleh Kingsley et al. \cite{kingsley2020improving}, telah berhasil meningkatkan kapasitas penyembunyian data secara signifikan hingga 450\%. Namun, implementasinya pada gambar grayscale membatasi aplikasinya secara luas \cite{soetarmono2012studi} karena sebagian besar gambar digital yang digunakan saat ini adalah gambar berwarna (RGB) yang memiliki tiga channel warna. Terdapat kebutuhan untuk mengembangkan metode steganografi berbasis kode yang dapat memanfaatkan potensi penuh gambar RGB untuk mencapai kapasitas penyembunyian yang lebih tinggi sambil tetap menjaga kualitas visual yang baik dan ketahanan terhadap deteksi.
}

\section{Objective and Hypotheses}
\justify{
    Tujuan utama penelitian ini adalah:
    \begin{enumerate}
        \item Mengembangkan teknik steganografi berbasis kode dengan multiple embedding yang
              dioptimalkan untuk gambar RGB, dengan target peningkatan kapasitas
              penyembunyian data minimal 50\% dibandingkan metode sebelumnya (Kingsley et al.
              \cite{kingsley2020improving}). Hipotesisnya adalah penggunaan Polar Codes pada
              gambar RGB akan memungkinkan peningkatan kapasitas yang lebih besar daripada
              aplikasi pada gambar grayscale.
        \item Mempertahankan kualitas visual gambar stego dengan nilai Peak Signal-to-Noise
              Ratio (PSNR) di atas 40 dB. Hipotesisnya adalah peningkatan kapasitas melalui
              multiple embedding dan Polar Codes tidak akan secara signifikan menurunkan
              kualitas visual gambar \cite{nasution2018image}.
    \end{enumerate}
    Metode yang diusulkan, yaitu multiple embedding dengan Polar Codes, dipilih karena potensi Polar Codes dalam mencapai kapasitas Shannon dengan kompleksitas yang relatif rendah dan kemampuannya dalam koreksi kesalahan, yang diharapkan dapat menyeimbangkan antara kapasitas dan kualitas visual pada gambar RGB.
}

\section{Research Method}
\subsection{Method requirement specification }
\justify{
    Kebutuhan utama metode yang dikembangkan adalah:
    \begin{enumerate}
        \item Mampu meningkatkan kapasitas penyembunyian data minimal 50\% dibandingkan
              metode Kingsley et al. \cite{kingsley2020improving}.
        \item Menjaga kualitas visual gambar stego dengan PSNR di atas 40 dB.
        \item Mengimplementasikan algoritma penyisipan multiple embedding yang efektif untuk
              gambar berwarna RGB.
    \end{enumerate}
    Sistem harus dapat memproses gambar input (secret image) dan gambar penampung (cover image) dalam format RGB, melakukan proses encoding/embedding dan decoding/extraction, serta tahan terhadap serangan umum.
}

\subsection{Design and Implementation of the proposed method }
\justify{
    Desain metode yang diusulkan melibatkan beberapa tahapan kunci seperti yang diilustrasikan pada Gambar 1 dalam proposal. Proses utamanya adalah menyisipkan bit-bit rahasia dari 'Secret Image' ke dalam 'Cover Color Image' menggunakan Polar Codes. Polar Codes digunakan untuk mengkodekan data rahasia sebelum penyisipan untuk meningkatkan reliabilitas dan mencapai kapasitas yang tinggi dengan kompleksitas $O(N \log N)$. Posisi piksel untuk penyisipan ditentukan secara acak namun deterministik berdasarkan seed. Selain itu, digunakan mekanisme Key Trace yang disisipkan ke dalam 'Agreed Image' untuk menghasilkan 'Agreed Stego', serta skema pembagian kunci (Shamir's Secret Sharing) di mana 'Share 2' disisipkan ke 'Stego Image' dan 'Share 1' disimpan terpisah. Implementasi dilakukan menggunakan Python dengan library OpenCV dan NumPy, mendukung gambar RGB dan grayscale untuk perbandingan. Fokus implementasi adalah pada mekanisme multiple embedding, adaptasi Polar Codes untuk RGB, optimasi kapasitas ($>$50\%), dan preservasi kualitas (PSNR $>$ 40 dB).
}

\subsection{Experimental results }
\justify{
    Tahap eksperimen dirancang untuk mengevaluasi tiga aspek utama: Imperceptibility, Robustness, dan Capacity, seperti pada Gambar 2 proposal. Imperceptibility diukur menggunakan Mean Square Error (MSE) dan Peak Signal-to-Noise Ratio (PSNR) untuk membandingkan kualitas visual gambar stego dengan gambar asli. Robustness diuji terhadap berbagai serangan seperti Noise (Gaussian \cite{selamiEffectGaussian}, Salt \& Pepper \cite{kaisar2008salt}, Speckle \cite{mansourpour2006effects}), Cropping, Scratching (Single \& Multiple), dan Kompresi JPEG \cite{reddy2015medical}, dengan mengukur PSNR gambar rahasia yang berhasil dipulihkan. Capacity dievaluasi dengan mengukur rasio bit rahasia yang dapat disematkan terhadap ukuran gambar cover (bpp), menggunakan variasi ukuran secret image dan karakteristik gambar cover. Hasil eksperimen metode yang diusulkan (Polar Codes dengan multiple embedding) akan dibandingkan dengan metode sebelumnya (Reed-Muller code dengan multiple embedding). Saat ini, pengujian awal sedang dipersiapkan seiring dengan penyelesaian tahap implementasi.
}

\subsection{Analysis of the experimental results}
\justify{
    Analisis hasil eksperimen akan difokuskan pada perbandingan kuantitatif antara metode yang diusulkan (Polar Codes) dan metode acuan (Reed-Muller). Metrik utama yang akan dianalisis adalah:
    \begin{itemize}
        \item Imperceptibility: Nilai MSE dan PSNR akan dihitung dan dibandingkan. PSNR yang
              lebih tinggi (di atas 40 dB) dan MSE yang lebih rendah menunjukkan kualitas
              visual yang lebih baik.
        \item Robustness: Nilai PSNR dari gambar rahasia yang dipulihkan setelah berbagai
              serangan akan dianalisis untuk mengukur ketahanan metode. Kemampuan koreksi
              kesalahan (ECC) dari kode yang digunakan (Polar Codes vs Reed-Muller) akan
              dievaluasi berdasarkan tingkat keberhasilan pemulihan data.
        \item Capacity: Kapasitas penyisipan (EC) dalam bit per pixel (bpp) akan dihitung
              menggunakan persamaan (6) dan dibandingkan antar metode. Analisis akan mencakup
              pengaruh variasi histogram gambar dan ukuran secret image terhadap kapasitas.
    \end{itemize}
    Analisis statistik seperti uji-t atau ANOVA mungkin digunakan untuk mengevaluasi signifikansi perbedaan kinerja antar metode.
}

\section{Schedule Realization}

\begin{table}[h]
    \centering
    \begin{tabular}{|l|c|c|c|c|l|l|}
        \hline
        \rowcolor[HTML]{FFFFFF}
        \multicolumn{1}{|c|}{\cellcolor[HTML]{FFFFFF}\multirow{2}{*}{Activity}} & \multicolumn{2}{c|}{\cellcolor[HTML]{FFFFFF}Schedule} & \multicolumn{2}{c|}{\cellcolor[HTML]{FFFFFF}Realization} & \multicolumn{1}{c|}{\cellcolor[HTML]{FFFFFF}\multirow{2}{*}{Output Target}} & \multicolumn{1}{c|}{\cellcolor[HTML]{FFFFFF}\multirow{2}{*}{Realization}}                                                                                                 \\ \cline{2-5}
        \rowcolor[HTML]{FFFFFF}
        \multicolumn{1}{|c|}{\cellcolor[HTML]{FFFFFF}}                          & \multicolumn{1}{c|}{\cellcolor[HTML]{FFFFFF}Start}    & \multicolumn{1}{c|}{\cellcolor[HTML]{FFFFFF}End}         & \multicolumn{1}{c|}{\cellcolor[HTML]{FFFFFF}Start}                          & \multicolumn{1}{c|}{\cellcolor[HTML]{FFFFFF}End}                          & \multicolumn{1}{c|}{\cellcolor[HTML]{FFFFFF}} & \multicolumn{1}{c|}{\cellcolor[HTML]{FFFFFF}} \\ \hline
        \rowcolor[HTML]{EFEFEF}
        Literature study                                                        & Sem 1                                                 & Sem 1                                                    & Sem 1                                                                       & Sem 1                                                                     & Proposal Bab 1, 2                             & Selesai                                       \\ \hline
        Problem identification                                                  & Sem 1                                                 & Sem 1                                                    & Sem 1                                                                       & Sem 1                                                                     & Proposal Bab 3                                & Selesai                                       \\ \hline
        \rowcolor[HTML]{EFEFEF}
        Contribution formulation                                                & Sem 1                                                 & Sem 1                                                    & Sem 1                                                                       & Sem 1                                                                     & Proposal Bab 4                                & Selesai                                       \\ \hline
        Hypothesis formulation                                                  & Sem 1                                                 & Sem 1                                                    & Sem 1                                                                       & Sem 1                                                                     & Proposal Bab 4                                & Selesai                                       \\ \hline
        \rowcolor[HTML]{EFEFEF}
        Proposal                                                                & Sem 1                                                 & Sem 1                                                    & Sem 1                                                                       & Sem 1                                                                     & Dokumen Proposal                              & Selesai                                       \\ \hline
        Design Encoding Process                                                 & Sem 2                                                 & Sem 2                                                    & Sem 2                                                                       & Sem 2                                                                     & Desain Algoritma                              & Selesai                                       \\ \hline
        \rowcolor[HTML]{EFEFEF}
        Implement embedding schema                                              & Sem 2                                                 & Sem 3                                                    & Sem 2                                                                       & \blue Sem 3                                                               & Kode Program (Embedding)                      & \blue On Progress                             \\ \hline
        Design decoding process                                                 & Sem 2                                                 & Sem 2                                                    & Sem 2                                                                       & \blue Sem 2                                                               & Desain Algoritma                              & \blue On Progress                             \\ \hline
        \rowcolor[HTML]{EFEFEF}
        Design sharing schema                                                   & Sem 2                                                 & Sem 3                                                    & Sem 2                                                                       & \blue Sem 3                                                               & Kode Program (Sharing)                        & \blue On Progress                             \\ \hline
        Imperceptibility testing                                                & Sem 3                                                 & Sem 3                                                    & Sem 3                                                                       & -                                                                         & Hasil Uji PSNR/MSE                            & Belum Mulai                                   \\ \hline
        \rowcolor[HTML]{EFEFEF}
        Robustness testing                                                      & Sem 3                                                 & Sem 4                                                    & Sem 3                                                                       & -                                                                         & Hasil Uji Serangan                            & Belum Mulai                                   \\ \hline
        Capacity testing                                                        & Sem 3                                                 & Sem 4                                                    & Sem 3                                                                       & -                                                                         & Hasil Uji Kapasitas                           & Belum Mulai                                   \\ \hline
        \rowcolor[HTML]{EFEFEF}
        Thesis draft                                                            & Sem 4                                                 & Sem 4                                                    & Sem 4                                                                       & -                                                                         & Draft Tesis Lengkap                           & Belum Mulai                                   \\ \hline
    \end{tabular}
    \caption{Realisasi Jadwal per 18 April 2025}
\end{table}

% Tambahkan bagian Daftar Pustaka
\newpage % Mulai halaman baru untuk daftar pustaka jika diinginkan
\begin{thebibliography}{99} % Angka 99 adalah placeholder untuk label terpanjang

    \bibitem{pujianto2021uji}
    Ikwan Pujianto. "Uji Ketahanan Citra Digital Terhadap Manipulasi Robustness Pada Steganography". \textit{Jurnal Informatika dan Rekayasa Perangkat Lunak} 2.1 (2021), pp. 16-27.

    \bibitem{siaulhak2023sistem}
    Siaulhak Siaulhak, Safwan Kasma et al. "Sistem Pengiriman File Menggunakan Steganografi Pengolahan Citra Digital Berbasis Matriks Laboratory". \textit{BANDWIDTH: Journal of Informatics and Computer Engineering} 1.2 (2023), pp. 75-81.

    \bibitem{soetarmono2012studi}
    Anggya ND Soetarmono. "Studi Mengenai Aplikasi Steganografi Camouflage". \textit{Teknika} 1.1 (2012), pp. 55-65.

    \bibitem{fikri2022optimasi}
    Muhammad Alfin Fikri and FX Ferdinandus. "Optimasi Teknik Steganografi Amelsbr Pada Empat Bit Terakhir Dengan Cover Image Berwarna". \textit{Antivirus: Jurnal Ilmiah Teknik Informatika} 16.1 (2022), pp. 25-38.

    \bibitem{kingsley2020improving}
    Katandawa Alex Kingsley and Ari Moesriami Barmawi. "Improving Data Hiding Capacity in Code Based Steganography using Multiple Embedding." In: (). % Mungkin perlu dilengkapi info publikasi jika ada

    \bibitem{andika2020modifikasi}
    Dwi Andika and Dedi Darwis. "Modifikasi Algoritma Gifshuffle Untuk Peningkatan Kualitas Citra Pada Steganografi". \textit{Jurnal Ilmiah Infrastruktur Teknologi Informasi} 1.2 (2020), pp. 19-23.

    \bibitem{susanto2020kombinasi}
    Ajib Susanto and Ibnu Utomo Wahyu Mulyono. "Kombinasi LSB-RSA untuk Peningkatan Imperceptibility pada Kripto-Stegano Gambar RGB". In: (2020). % Mungkin perlu dilengkapi info publikasi jika ada

    \bibitem{basri2021penerapan}
    Muh Basri and Muhammad Fadhlil Gushari. "Penerapan Steganografi Gambar Berwarna pada Delapan Image Cover Menggunakan Metode LSB". \textit{Jurnal Sintaks Logika} 1.3 (2021), pp. 153-158.

    \bibitem{wiranata2021aplikasi}
    Ade Davy Wiranata and Rima Tamara Aldisa. "Aplikasi Steganografi Menggunakan Least Significant Bit (LSB) dengan Enkripsi Caesar Chipper dan Rivest Code 4 (RC4) Menggunakan Bahasa Pemrograman JAVA". \textit{Jurnal JTIK (Jurnal Teknologi Informasi dan Komunikasi)} 5.3 (2021), pp. 277-281.

    \bibitem{abdillah2023implementasi}
    Muhammad Oemar Abdillah, Ogie Ariansah Pane and Farhan Rusdy Asyhary Lubis. "Implementasi Keamanan Aset Informasi Steganografi Menggunakan Metode Least Significant Bit (LSB)". \textit{Jurnal Sains dan Teknologi (JSIT)} 3.1 (2023), pp. 40-46.

    \bibitem{juneja2013improved}
    Mamta Juneja and Parvinder S Sandhu. "An improved LSB based steganography technique for RGB color images". \textit{International journal of computer and communication engineering} 2.4 (2013), p. 513.

    \bibitem{nasution2018image}
    AB Nasution, S Efendi and S Suwilo. "Image steganography in securing sound file using arithmetic coding algorithm, triple data encryption standard (3DES) and modified least significant bit (MLSB)". \textit{Journal of Physics: Conference Series}. Vol. 1007. 1. IOP Publishing. 2018, p. 012010.

    \bibitem{selamiEffectGaussian}
    Ameen Mohammed Abd-Alsalam Selami and Ahmed Freidoon Fadhil. "A Study of the Effects of Gaussian Noise on Image". In: (). % Mungkin perlu dilengkapi info publikasi jika ada

    \bibitem{kaisar2008salt}
    Shakair Kaisar and Jubayer AI Mahmud. "Salt and Pepper Noise Detection and removal by Tolerance based selective Arithmetic Mean Filtering Technique for image restoration". \textit{International Journal of computer science and network security} 8.6 (2008), pp. 271-278.

    \bibitem{mansourpour2006effects}
    Mostafa Mansourpour, MA Rajabi and JAR Blais. "Effects and performance of speckle noise reduction filters on active radar and SAR images". In: Proc. Isprs. Vol. 36. 1. 2006, W41. % Mungkin perlu dilengkapi info publikasi jika ada

    \bibitem{reddy2015medical}
    Mr Venugopal Reddy CH et al. "Medical image watermarking schemes against salt and pepper noise attack". \textit{International Journal of Bio-Science and Bio-Technology} 7.6 (2015), pp. 55-64.

\end{thebibliography}

%\section{Summary}
%\supervisorcomments
\end{document}